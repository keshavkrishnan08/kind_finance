% ============================================================================
% Non-Equilibrium Koopman-Thermodynamic Neural Decomposition
% for Financial Market Dynamics
%
% Target: Physical Review E
% ============================================================================

\documentclass[aps,pre,twocolumn,superscriptaddress,showpacs,floatfix]{revtex4-2}

% ---- Packages ----
\usepackage{amsmath,amssymb,amsfonts}
\usepackage{graphicx}
\usepackage{bm}
\usepackage{hyperref}
\usepackage{xcolor}
\usepackage{algorithm}
\usepackage{algpseudocode}
\usepackage{booktabs}
\usepackage{dcolumn}
\usepackage{natbib}
\usepackage{physics}
\usepackage{amsthm}
\newtheorem{proposition}{Proposition}

% ---- Graphics path ----
\graphicspath{{../outputs/figures/}}

% ---- Custom macros ----
\newcommand{\Koop}{\mathcal{K}}
\newcommand{\Trans}{\mathcal{T}}
\newcommand{\Hilb}{\mathcal{H}}
\newcommand{\R}{\mathbb{R}}
\newcommand{\C}{\mathbb{C}}
\newcommand{\E}{\mathbb{E}}
\newcommand{\KL}{\mathrm{KL}}
\newcommand{\ep}{\dot{S}}
\newcommand{\Frob}[1]{\left\lVert #1 \right\rVert_{\mathrm{F}}}

\begin{document}

% ============================================================================
\title{Non-Equilibrium Koopman-Thermodynamic Neural Decomposition\\
for Financial Market Dynamics}

\author{Keshav Krishnan}
\affiliation{Independent Researcher}

\date{\today}

\pacs{89.65.Gh, 05.70.Ln, 02.50.Ga, 05.45.Tp}
% 89.65.Gh -- Economics; econophysics, financial markets, business and management
% 05.70.Ln -- Nonequilibrium and irreversible thermodynamics
% 02.50.Ga -- Markov processes
% 05.45.Tp -- Time series analysis

% ============================================================================
\begin{abstract}
We introduce a spectral framework for quantifying irreversibility in driven
complex systems, uniting Koopman operator theory with non-equilibrium
statistical mechanics.  A dual-lobe variational neural network learns the
left and right eigenfunctions of a non-self-adjoint Koopman operator from
time-series data, yielding complex eigenvalues that encode oscillatory
probability currents and broken detailed balance.  The entropy production
rate decomposes into per-mode contributions
$\dot{S}_k = \omega_k^2 A_k / \gamma_k$, and a pointwise irreversibility field
$I(\mathbf{x})$ localizes non-equilibrium behavior in state space.
Applied to U.S.\ equity markets (2007--2026), 8 of 10 resolved Koopman
modes are complex---direct spectral evidence of rotational probability
currents.  Iterative amplitude-adjusted Fourier transform (IAAFT)
surrogate testing establishes the irreversibility at
$p < 10^{-4}$ with Cohen's $d = 33.0$, ruling out linear-autocorrelation
artifacts.  The framework outperforms hidden Markov, DMD, GARCH, and PCA
baselines in NBER recession detection (F1~$= 0.50$ vs.\ $\leq 0.37$),
and all results reproduce over five random seeds
(multiasset NBER accuracy $0.854 \pm 0.005$).
\end{abstract}

\maketitle

% ============================================================================
% I. INTRODUCTION
% ============================================================================
\section{Introduction}\label{sec:intro}

Quantifying irreversibility---the extent to which a system's dynamics
violate time-reversal symmetry---is a central problem in non-equilibrium
statistical mechanics~\cite{seifert2012stochastic,ciliberto2017experiments}.
For physical systems, entropy production provides the fundamental measure of
irreversibility, and recent advances in machine learning have enabled its
estimation from trajectory data in biological~\cite{battle2016broken,
li2019quantifying}, colloidal~\cite{seara2021irreversibility}, and
synthetic~\cite{otsubo2020estimating,kim2024fdivergence} systems.
Financial markets offer a compelling arena for these methods: they are
driven far from equilibrium by information asymmetry, heterogeneous agent
strategies, and regulatory feedback
loops~\cite{bouchaud2003theory,mantegna1999introduction}, producing
well-documented signatures of time-reversal asymmetry including
fat-tailed return distributions, volatility clustering, and the leverage
effect~\cite{cont2001empirical,zumbach2009time}.

Despite this, the equilibrium paradigm has dominated quantitative finance.
The geometric Brownian motion underlying the Black-Scholes
framework~\cite{black1973pricing} and the efficient market
hypothesis~\cite{fama1970efficient} both assume time-reversible price
dynamics---equivalent to detailed balance in the language of statistical
mechanics. Extensions such as GARCH
processes~\cite{bollerslev1986generalized}, stochastic volatility
models~\cite{heston1993closed}, and Markov regime-switching
models~\cite{hamilton1989new,ang2002regime} introduce richer dynamics but
do not provide a systematic framework for quantifying \emph{how far} the
market deviates from equilibrium or for decomposing that deviation into
dynamically interpretable spectral modes.

We note that the application of physics methods to economic systems
has been subject to methodological
critiques~\cite{gallegati2006worrying}, particularly regarding
insufficient engagement with the economics literature, lack of
statistical rigor, and overreliance on physical analogies without formal
justification. We address these concerns explicitly: our framework is
grounded in rigorous operator-theoretic results (not physical analogy),
we benchmark against standard econometric baselines
(HMM~\cite{hamilton1989new}, GARCH~\cite{bollerslev1986generalized}),
and we provide comprehensive statistical validation including
bootstrap confidence intervals, permutation tests, and
Chapman-Kolmogorov consistency checks.

Recent work has begun to forge rigorous connections between
non-equilibrium thermodynamics and finance.
Ducuara \emph{et al.}~\cite{ducuara2023maxwell} proved that Crooks'
fluctuation relations map onto expected utility theory, establishing
a formal mathematical bridge (not merely an analogy) between
stochastic thermodynamics and financial decision theory. Rold{\'a}n
and Parrondo~\cite{roldan2010estimating} developed methods for
estimating dissipation from single stationary trajectories.
Zumbach~\cite{zumbach2009time} demonstrated empirically that financial
time series violate time-reversal invariance at multiple scales. In the
broader context of non-equilibrium estimation, machine learning approaches
have achieved state-of-the-art entropy production estimates in physical
systems~\cite{otsubo2020estimating,kim2024fdivergence}. We build on
these developments by providing a \emph{spectral} decomposition of
irreversibility.

In stochastic thermodynamics, the departure from equilibrium is
quantified by the \textit{entropy production rate}
\begin{equation}\label{eq:ep_intro}
  \ep = \lim_{\tau\to0}\frac{1}{\tau}
    D_{\KL}\!\left[P(\mathbf{x}_{t+\tau}|\mathbf{x}_t)
    \,\middle\|\, P_{\mathrm{rev}}(\mathbf{x}_{t+\tau}|\mathbf{x}_t)\right],
\end{equation}
where $D_{\KL}$ is the Kullback-Leibler divergence between the forward and
time-reversed transition densities~\cite{esposito2010three}. A vanishing
$\ep$ recovers detailed balance; a positive $\ep$ signals irreversibility.
For financial markets, $\ep > 0$ encodes the ``arrow of time'' imprinted by
informed trading, momentum strategies, and central bank interventions.

The \textit{Koopman operator}~\cite{koopman1931hamiltonian,mezic2005spectral}
provides a complementary spectral lens. Acting on observable functions
$f:\Omega\to\R$, the Koopman operator $\Koop^\tau$ propagates expectations
forward in time:
\begin{equation}\label{eq:koopman_intro}
  [\Koop^\tau f](\mathbf{x}) = \E[f(\mathbf{x}_{t+\tau}) | \mathbf{x}_t = \mathbf{x}].
\end{equation}
Its eigenvalues $\lambda_k$ encode timescales ($t_k = -\tau/\ln|\lambda_k|$)
and oscillation frequencies ($\omega_k = \arg(\lambda_k)/\tau$), while
eigenfunctions $\psi_k$ partition state space into dynamically coherent
regions~\cite{budivsic2012applied}. For reversible dynamics, $\Koop^\tau$ is
self-adjoint and all eigenvalues are real. \textit{Complex} eigenvalues---the
hallmark of non-equilibrium dynamics---encode rotational probability currents
that break time-reversal symmetry.

The deep learning revolution in molecular dynamics, initiated by
VAMPnets~\cite{mardt2018vampnets} and the variational approach for Markov
processes (VAMP)~\cite{wu2020variational}, demonstrated that neural networks
can learn Koopman eigenfunctions from trajectory data by maximizing a
variational score. However, the original VAMPnet framework was designed for
\emph{reversible} dynamics (shared encoder weights, real eigenvalues) and
applied exclusively to molecular systems satisfying detailed balance.

In this work, we bridge three fields---non-equilibrium statistical mechanics,
Koopman operator spectral theory, and deep learning---to construct a
framework purpose-built for \emph{non-reversible} financial market dynamics.
Our contributions are:

\begin{enumerate}
  \item \textbf{Non-equilibrium Koopman framework.}  A dual-lobe neural
    architecture with independent encoder parameters learns the left and
    right eigenfunctions of a non-self-adjoint Koopman operator, yielding
    complex eigenvalues that encode oscillatory probability currents.  We
    derive a per-mode entropy production decomposition
    $\dot{S}_k = \omega_k^2 A_k / \gamma_k$ from the Kullback-Leibler divergence
    between forward and time-reversed transition densities
    (Secs.~\ref{sec:architecture},~\ref{sec:entropy}).

  \item \textbf{Pointwise irreversibility diagnostic.}  We construct an
    irreversibility field
    $I(\mathbf{x}) = \sum_k \sigma_k |u_k(\mathbf{x}) - v_k(\mathbf{x})|^2$
    and prove that $I(\mathbf{x}) = 0$ if and only if the projected
    dynamics satisfy detailed balance, providing state-space-resolved
    information about non-equilibrium behavior that is inaccessible to
    global entropy production estimates
    (Sec.~\ref{sec:irreversibility}).

  \item \textbf{Rigorous empirical validation.}  We validate on
    non-reversible Langevin dynamics, apply to 19~years of U.S.\ equity
    data with IAAFT surrogate testing ($p < 10^{-4}$, Cohen's
    $d = 33.0$), and benchmark against HMM, DMD, GARCH, and PCA
    baselines (Secs.~\ref{sec:synthetic}--\ref{sec:results}).
\end{enumerate}

The remainder of this paper is organized as follows.
Section~\ref{sec:theory} develops the theoretical framework connecting
Koopman spectral theory to non-equilibrium thermodynamics.
Section~\ref{sec:ml_foundations} presents the machine learning formulation,
including the VAMP variational principle and network architecture.
Section~\ref{sec:methods} describes the data pipeline, training procedure,
and analysis tools. Sections~\ref{sec:synthetic} and~\ref{sec:results}
present results on synthetic and financial data, respectively.
Section~\ref{sec:discussion} discusses implications and limitations, and
Section~\ref{sec:conclusion} concludes.


% ============================================================================
% II. THEORETICAL FRAMEWORK
% ============================================================================
\section{Theoretical Framework}\label{sec:theory}

\subsection{Koopman operator for stochastic dynamics}
\label{sec:koopman_theory}

Consider a discrete-time stochastic process $\{\mathbf{x}_t\}_{t \geq 0}$
on a state space $\Omega \subseteq \R^d$ with transition kernel
$p_\tau(\mathbf{y}|\mathbf{x})$. The \textit{Koopman operator}
$\Koop^\tau : \Hilb \to \Hilb$ acts on square-integrable observables
$f \in L^2(\Omega, \mu)$ as
\begin{equation}\label{eq:koopman}
  [\Koop^\tau f](\mathbf{x})
  = \int_\Omega p_\tau(\mathbf{y}|\mathbf{x})\, f(\mathbf{y})\,
    \mathrm{d}\mathbf{y}
  = \E[f(\mathbf{x}_{t+\tau}) | \mathbf{x}_t = \mathbf{x}].
\end{equation}
Its adjoint with respect to the stationary measure $\mu$ is the
\textit{transfer operator} (Perron-Frobenius operator) $\Trans^\tau$, which
propagates densities:
\begin{equation}
  \int_A [\Trans^\tau \rho](\mathbf{y})\,\mathrm{d}\mathbf{y}
  = \int_\Omega \rho(\mathbf{x})\,
    p_\tau(A|\mathbf{x})\,\mathrm{d}\mathbf{x}.
\end{equation}
The duality $\langle \Koop^\tau f, \rho \rangle_\mu
= \langle f, \Trans^\tau \rho \rangle_\mu$ connects spectral properties of
$\Koop^\tau$ to the metastable decomposition of state space.

For a process satisfying \textit{detailed balance},
\begin{equation}\label{eq:detailed_balance}
  \mu(\mathbf{x})\,p_\tau(\mathbf{y}|\mathbf{x})
  = \mu(\mathbf{y})\,p_\tau(\mathbf{x}|\mathbf{y}),
\end{equation}
$\Koop^\tau$ is self-adjoint on $L^2(\Omega,\mu)$. Its eigenvalues
$\{\lambda_k\}$ are real with $|\lambda_k| \leq 1$, and the
eigenfunctions $\{\psi_k\}$ form a complete orthonormal basis. The spectral
decomposition reads
\begin{equation}\label{eq:spectral_reversible}
  p_\tau(\mathbf{y}|\mathbf{x})
  = \mu(\mathbf{y}) \sum_{k=0}^{\infty}
    \lambda_k\, \psi_k(\mathbf{x})\, \psi_k(\mathbf{y}).
\end{equation}

\subsection{Breaking detailed balance: non-reversible dynamics}
\label{sec:breaking_db}

When detailed balance is broken, $\Koop^\tau$ is no longer self-adjoint.
Its eigenvalues may be \emph{complex}, $\lambda_k = |\lambda_k|\,
e^{i\omega_k \tau}$, and the left and right eigenfunctions are distinct:
\begin{align}
  \Koop^\tau u_k &= \lambda_k\, u_k
    \quad \text{(right eigenfunctions)}, \label{eq:right_eig} \\
  \Koop^{\tau\dagger} v_k &= \bar{\lambda}_k\, v_k
    \quad \text{(left eigenfunctions)}. \label{eq:left_eig}
\end{align}
The biorthogonality condition $\langle v_j, u_k \rangle_\mu = \delta_{jk}$
replaces the orthogonality of the reversible case. The transition density
generalizes to
\begin{equation}\label{eq:spectral_nonreversible}
  p_\tau(\mathbf{y}|\mathbf{x})
  = \mu(\mathbf{y}) \sum_{k=0}^{\infty}
    \lambda_k\, u_k(\mathbf{x})\, v_k(\mathbf{y}).
\end{equation}
The imaginary parts $\omega_k = \arg(\lambda_k)/\tau$ encode
\textit{oscillatory modes}---rotational probability currents in state
space that have no counterpart in equilibrium.

The \textit{singular value decomposition} of $\Koop^\tau$ provides an
alternative, unconditionally real decomposition:
\begin{equation}\label{eq:svd_koopman}
  \Koop^\tau = \sum_{k=0}^{K-1} \sigma_k\, |u_k\rangle \langle v_k|,
\end{equation}
where $\sigma_k \geq 0$ are the singular values and $|u_k\rangle$,
$|v_k\rangle$ are the right and left singular vectors, respectively. For
reversible dynamics, $\sigma_k = |\lambda_k|$ and $u_k = v_k = \psi_k$.

\subsection{Spectral gap and regime persistence}
\label{sec:spectral_gap}

The \textit{spectral gap} of the Koopman operator,
\begin{equation}\label{eq:spectral_gap}
  \Delta = \frac{|\operatorname{Re}(\ln \lambda_2)|}{\tau},
\end{equation}
where $\lambda_2$ is the subdominant eigenvalue (second-largest in modulus),
governs the rate of approach to the stationary distribution. The gap provides
a rigorous lower bound on the \textit{regime persistence time}:
\begin{equation}\label{eq:persistence}
  T_{\mathrm{persist}} \geq \frac{1}{\Delta}.
\end{equation}
This inequality connects to Kramers' escape theory for
diffusion in a double-well potential~\cite{kramers1940brownian}: for
overdamped Langevin dynamics in a potential $V(\mathbf{x})$ with barrier
height $\Delta V$ and diffusion coefficient $D$, the dominant non-trivial
decay rate is
\begin{equation}\label{eq:kramers}
  \gamma_{\mathrm{Kramers}}
  = \frac{\sqrt{V''(\mathbf{x}_{\min})\,|V''(\mathbf{x}_{\mathrm{bar}})|}}
    {2\pi}\, \exp\!\left(-\frac{\Delta V}{D}\right),
\end{equation}
and $\Delta = \gamma_{\mathrm{Kramers}}$ to leading order.

For non-reversible dynamics, the bound~\eqref{eq:persistence} remains valid
as a consequence of the spectral mapping theorem: the slowest-decaying
mode sets a lower bound on mixing time regardless of
reversibility~\cite{hwang2005accelerating}.  Non-conservative forces
can accelerate mixing beyond this
bound~\cite{rey-bellet2006open}, so $1/\Delta$ is conservative.

\paragraph{Finite-dimensional Galerkin projection.}
The VAMP framework (Sec.~\ref{sec:vamp}) yields a $K \times K$ matrix
$\mathbf{K}$ that is the optimal rank-$K$ Galerkin projection of the
infinite-dimensional Koopman operator onto the learned basis
functions~\cite{wu2020variational}.  Its eigenvalues approximate the
$K$ dominant Koopman eigenvalues, and the VAMP-2 score
$\mathcal{R}_2 = \sum_k \sigma_k^2$ provides a variational certificate
of approximation quality: higher $\mathcal{R}_2$ implies that a larger
fraction of the operator's spectral content is resolved.  The unresolved
modes ($k > K$) contribute to residual autocorrelation in the projected
dynamics; we assess this via Ljung-Box tests (Sec.~\ref{sec:stat_results})
and interpret significant residuals as evidence that $K$ modes provide a
low-rank approximation, not an exact representation, of the full dynamics.

\subsection{Entropy production from the Koopman spectrum}
\label{sec:entropy}

The entropy production rate for a stationary Markov process measures the
irreversibility of the dynamics~\cite{seifert2012stochastic}:
\begin{equation}\label{eq:ep_kl}
  \ep = \frac{1}{\tau}
    D_{\KL}\!\left[P_\tau^{\mathrm{fwd}}(\mathbf{x},\mathbf{y})
    \,\middle\|\, P_\tau^{\mathrm{rev}}(\mathbf{x},\mathbf{y})\right] \geq 0,
\end{equation}
where $P_\tau^{\mathrm{fwd}}(\mathbf{x},\mathbf{y}) =
\mu(\mathbf{x})\,p_\tau(\mathbf{y}|\mathbf{x})$ is the forward path
measure and $P_\tau^{\mathrm{rev}}(\mathbf{x},\mathbf{y}) =
\mu(\mathbf{y})\,p_\tau(\mathbf{x}|\mathbf{y})$ is the time-reversed
measure. Detailed balance holds if and only if $\ep = 0$.

We now derive the per-mode decomposition.
The time-reversed transition density follows from Bayes' theorem and
stationarity ($\int \mu(\mathbf{x})\,p_\tau(\mathbf{y}|\mathbf{x})\,
\mathrm{d}\mathbf{x} = \mu(\mathbf{y})$):
\begin{equation}\label{eq:reverse_kernel}
  \tilde{p}_\tau(\mathbf{y}|\mathbf{x})
  = \frac{\mu(\mathbf{y})\,p_\tau(\mathbf{x}|\mathbf{y})}{\mu(\mathbf{x})}.
\end{equation}
Substituting the biorthogonal expansion~\eqref{eq:spectral_nonreversible}
into~\eqref{eq:reverse_kernel} and using
$p_\tau(\mathbf{x}|\mathbf{y}) = \mu(\mathbf{x})\sum_k \lambda_k\,
u_k(\mathbf{y})\,v_k(\mathbf{x})$ gives
\begin{equation}\label{eq:reverse_spectral}
  \tilde{p}_\tau(\mathbf{y}|\mathbf{x})
  = \mu(\mathbf{y}) \sum_{k=0}^{\infty}
    \lambda_k\, v_k(\mathbf{x})\, u_k(\mathbf{y}),
\end{equation}
which differs from the forward kernel~\eqref{eq:spectral_nonreversible}
by the interchange $u_k \leftrightarrow v_k$.  For reversible dynamics
$u_k = v_k$ and the two kernels coincide; for non-reversible dynamics
the eigenvector asymmetry generates a nonvanishing log-ratio.

The entropy production~\eqref{eq:ep_kl} involves the expectation of
$\ln[p_\tau/\tilde{p}_\tau]$ under the forward measure.  Define the
forward and reversed spectral sums (excluding the trivial $k=0$ mode):
\begin{equation}
  f(\mathbf{x},\mathbf{y}) = \sum_{k \geq 1} \lambda_k\, u_k(\mathbf{x})\,
    v_k(\mathbf{y}), \quad
  \tilde{f}(\mathbf{x},\mathbf{y}) = \sum_{k \geq 1} \lambda_k\,
    v_k(\mathbf{x})\, u_k(\mathbf{y}).
\end{equation}
Setting $\Delta f = f - \tilde{f}$, the log-ratio
$\ln[(1+f)/(1+\tilde{f})]$ can be expanded to leading order
in $\Delta f/(1+\tilde{f})$:
\begin{equation}\label{eq:log_expand}
  \ln\frac{p_\tau}{\tilde{p}_\tau}
  = \frac{\Delta f}{1 + \tilde{f}}
    - \frac{(\Delta f)^2}{2(1 + \tilde{f})^2} + O(\Delta f^3).
\end{equation}
The first-order term vanishes upon averaging against the stationary
measure by the biorthogonality $\langle v_j, u_k \rangle_\mu =
\delta_{jk}$.  The leading nonvanishing contribution is therefore
\emph{second-order} in the forward-backward asymmetry:
\begin{equation}\label{eq:ep_secondorder}
  \ep = \frac{1}{2\tau}
    \left\langle \frac{(\Delta f)^2}{(1+\tilde{f})^2}
    \right\rangle_{\!\!P^{\mathrm{fwd}}} + O(\Delta f^4).
\end{equation}
To obtain a mode-resolved expression, we evaluate~\eqref{eq:ep_secondorder}
in the finite-dimensional Koopman matrix representation.
Decompose the whitened Koopman matrix (Eq.~\ref{eq:whitened_K}) as
$\mathbf{K} = \mathbf{K}_S + \mathbf{K}_A$, where
$\mathbf{K}_S = (\mathbf{K} + \mathbf{K}^T)/2$ is the symmetric
(equilibrium) part and $\mathbf{K}_A = (\mathbf{K} - \mathbf{K}^T)/2$
is the antisymmetric part encoding non-equilibrium currents.
In the whitened basis the forward transition is represented by
$\mathbf{K}$ and the time-reversed transition by $\mathbf{K}^T$
(since time reversal interchanges left and right singular vectors,
which corresponds to transposition in the whitened frame).
The element-wise forward-backward difference is therefore
$K_{ij} - K_{ji} = 2(\mathbf{K}_A)_{ij}$, and the symmetrized
element is $\frac{1}{2}(K_{ij} + K_{ji}) = (\mathbf{K}_S)_{ij}$.

The log-ratio expansion~\eqref{eq:log_expand}, applied element-wise
to the $K \times K$ matrix, gives
\begin{equation}
  \ln\frac{K_{ij}}{K_{ji}}
  = \ln\frac{(\mathbf{K}_S)_{ij} + (\mathbf{K}_A)_{ij}}
           {(\mathbf{K}_S)_{ij} - (\mathbf{K}_A)_{ij}}
  \approx \frac{2(\mathbf{K}_A)_{ij}}{(\mathbf{K}_S)_{ij}}
\end{equation}
for $|(\mathbf{K}_A)_{ij}| \ll |(\mathbf{K}_S)_{ij}|$.
Summing the forward-weighted log-ratios and noting that the
first-order term $\sum_{ij} (\mathbf{K}_A)_{ij} = 0$ (antisymmetry),
the leading contribution to the entropy production rate is
\begin{equation}\label{eq:ep_perturbative}
  \ep = \frac{2}{\tau}\sum_{i,j}
    \frac{(\mathbf{K}_A)_{ij}^2}{(\mathbf{K}_S)_{ij}}
    + O\!\left(\|\mathbf{K}_A\|^4\right).
\end{equation}
Diagonalising $\mathbf{K}_S$ with eigenvalues $\mu_k$ (and
defining the continuous-time decay rates
$\gamma_k = -\ln\mu_k / \tau > 0$), the sum in~\eqref{eq:ep_perturbative}
reduces in the eigenbasis of $\mathbf{K}_S$ to
\begin{equation}\label{eq:ep_modesum}
  \ep = \frac{2}{\tau}\sum_{k=1}^{K}
    \frac{1}{\mu_k}\sum_{j}(\mathbf{K}_A)_{jk}^2
    = \frac{2}{\tau}\,
    \mathrm{tr}\!\left(\mathbf{K}_A^T\,
    \mathbf{K}_S^{-1}\,\mathbf{K}_A\right),
\end{equation}
where $\mathbf{K}_S^{-1} = \mathrm{diag}(1/\mu_k)$ in the eigenbasis.
The column norm $\sum_j (\mathbf{K}_A)_{jk}^2$ is the total
antisymmetric coupling of mode $k$; for a mode with complex
eigenvalue $\lambda_k = |\lambda_k|\,e^{i\omega_k\tau}$, this norm
is $\propto \omega_k^2\tau^2$ at leading order in the oscillation angle
$\omega_k\tau$.  Combining with $1/\mu_k = e^{\gamma_k\tau}
\approx 1 + \gamma_k\tau$ for moderate $\gamma_k\tau$, and absorbing
the eigenfunction spatial structure into the amplitude
$A_k = \langle |\psi_k|^2 \rangle_\mu$, the per-mode spectral entropy
production takes the form
\begin{equation}\label{eq:ep_mode}
  \dot{S}_k = \frac{\omega_k^2}{\gamma_k}\, A_k,
\end{equation}
where $\omega_k = \arg(\lambda_k)/\tau$ is the oscillation frequency
and $\gamma_k = -\ln|\lambda_k|/\tau > 0$ is the decay rate.
The total resolved spectral entropy production is
\begin{equation}\label{eq:ep_total}
  \ep_{\mathrm{spectral}} = \sum_{k=1}^{K}
    \frac{\omega_k^2}{\gamma_k}\, A_k.
\end{equation}
Equation~\eqref{eq:ep_mode} has three physically transparent factors:
(i)~$\omega_k^2$ encodes the quadratic KL-divergence scaling
from~\eqref{eq:ep_secondorder}---only oscillatory modes
($\omega_k \neq 0$) break time-reversal symmetry;
(ii)~$1/\gamma_k$ amplifies slowly decaying modes, which sustain
probability currents for longer before relaxing to equilibrium;
and (iii)~$A_k$ weights each mode by its spatial extent in the
stationary measure.

We stress two remaining limitations.
First, the sum runs over only the $K$ resolved Koopman modes;
unresolved higher-frequency modes ($k > K$) contribute additional
irreversibility that the spectral decomposition does not capture.
Because $\dot{S}_k \propto \omega_k^2$, these fast modes can carry a
substantial fraction of the total entropy production.
Second, the perturbative expansion~\eqref{eq:ep_perturbative} is valid
when $\|\mathbf{K}_A\|_F \ll \|\mathbf{K}_S\|_F$; for strongly
non-equilibrium dynamics, higher-order terms become relevant.
We verify self-consistency empirically through the entropy production
consistency loss (Sec.~\ref{sec:loss}).

\subsection{Irreversibility field}
\label{sec:irreversibility}

To localize broken detailed balance in state space, we construct the
\textit{irreversibility field} from the singular value decomposition
(SVD) of the Koopman operator.
Recall that the SVD~\eqref{eq:svd_koopman} provides the decomposition
$\Koop^\tau = \sum_k \sigma_k\,|u_k\rangle\langle v_k|$, where
$\{u_k\}$ and $\{v_k\}$ are the right and left singular vectors
(orthonormal bases of the range and domain, respectively) and
$\sigma_k \geq 0$ are the singular values.
These are distinct from the biorthogonal \emph{eigen}functions of
Sec.~\ref{sec:breaking_db}: singular vectors are always real-valued
and orthonormal, whereas eigenfunctions may be complex and satisfy
only biorthogonality.  We define
\begin{equation}\label{eq:irrev_field}
  I(\mathbf{x}) = \sum_{k=1}^{K} \sigma_k\, |u_k(\mathbf{x}) - v_k(\mathbf{x})|^2.
\end{equation}

\begin{proposition}\label{prop:irrev}
$I(\mathbf{x}) = 0$ for $\mu$-almost every $\mathbf{x}$ if and only if
$\Koop^\tau$ restricted to the $K$-dimensional learned subspace is
self-adjoint (i.e., the projected dynamics satisfy detailed balance).
\end{proposition}

\begin{proof}
(\emph{Forward.})
If $\Koop^\tau$ is self-adjoint on $L^2(\Omega,\mu)$, its SVD coincides
with the eigendecomposition: $\sigma_k = |\lambda_k|$ and $u_k = v_k$
for each $k$ (the left and right singular vectors are the orthonormal
eigenfunctions).  Thus every term
$|u_k(\mathbf{x}) - v_k(\mathbf{x})|^2 = 0$, giving $I(\mathbf{x}) = 0$.

(\emph{Backward.})
Suppose $I(\mathbf{x}) = 0$ $\mu$-a.e.  Since $\sigma_k > 0$ for each
retained mode, each summand must vanish: $u_k = v_k$ in
$L^2(\Omega,\mu)$ for $k = 1, \ldots, K$.  The SVD restricted to the
learned subspace is then
$\Koop^\tau\big|_K = \sum_{k=1}^K \sigma_k\,|u_k\rangle\langle u_k|$,
which is manifestly self-adjoint.  By the duality
condition~\eqref{eq:detailed_balance}, this implies that the
$K$-dimensional projected dynamics satisfy detailed balance.
\end{proof}

The magnitude of $I(\mathbf{x})$ at a given state quantifies the local
strength of non-equilibrium probability currents.
We emphasize that Proposition~\ref{prop:irrev} is a statement about the
\emph{projected} dynamics in the $K$-dimensional Koopman subspace;
$I(\mathbf{x}) = 0$ does not imply global detailed balance, since
unresolved modes may still carry irreversibility.

\subsection{Detailed balance violation and fluctuation theorems}
\label{sec:db_violation}

The asymmetry of the Koopman matrix provides a global measure of
detailed balance violation. For a finite-dimensional approximation
$\mathbf{K} \in \R^{d \times d}$, we define
\begin{equation}\label{eq:db_metric}
  \mathcal{D}
  = \frac{\Frob{\mathbf{K} - \mathbf{K}^T}}{\Frob{\mathbf{K}}},
\end{equation}
which vanishes for reversible dynamics ($\mathbf{K} = \mathbf{K}^T$) and
reaches its maximum when the symmetric part of $\mathbf{K}$ vanishes.

For finite-time trajectories, the fluctuation theorem constrains the
distribution of the entropy production. The Gallavotti-Cohen symmetry
function~\cite{gallavotti1995dynamical}
\begin{equation}\label{eq:gc_symmetry}
  \zeta(\dot{s})
  = \frac{1}{\tau}\ln\frac{P(+\dot{s})}{P(-\dot{s})}
\end{equation}
satisfies $\zeta(\dot{s}) = \dot{s}$ for systems obeying the steady-state
fluctuation theorem. Deviations from linearity indicate finite-size effects
or non-stationary contributions.

\subsection{Chapman-Kolmogorov consistency}
\label{sec:ck_theory}

A necessary condition for the learned Koopman operator to be
self-consistent is the Chapman-Kolmogorov equation: for any integer $n$,
\begin{equation}\label{eq:ck}
  \mathbf{K}(n\tau) = [\mathbf{K}(\tau)]^n.
\end{equation}
We test this by computing the Frobenius norm of the deviation
\begin{equation}\label{eq:ck_residual}
  \epsilon_{\mathrm{CK}}(n)
  = \Frob{[\mathbf{K}(\tau)]^n - \mathbf{K}_{\mathrm{direct}}(n\tau)}
\end{equation}
for $n = 2, 3, \ldots, n_{\max}$, where $\mathbf{K}_{\mathrm{direct}}(n\tau)$
is independently estimated from data pairs separated by $n\tau$. A
block-bootstrap null distribution provides $p$-values for each $n$.


\subsection{Applicability to driven complex systems}
\label{sec:applicability}

The Koopman--VAMP framework rests on two structural
assumptions whose validity for financial data requires justification.
\emph{Markovianity} is enforced via Takens delay
embedding~\cite{takens1981detecting} ($m = 5$, Sec.~\ref{sec:embedding})
and verified by the Chapman-Kolmogorov test (Sec.~\ref{sec:ck_theory}).
\emph{Local stationarity} is approximate: financial markets are globally
non-stationary, but the rolling-window analysis
(Sec.~\ref{sec:rolling_results}) refreshes the Koopman operator every
500~days, tracking spectral evolution explicitly.

We emphasize that the Koopman operator is a \emph{kinematic} object
defined for any stochastic dynamics, with or without a
Hamiltonian~\cite{budivsic2012applied,brunton2022modern}.
The thermodynamic quantities we compute---entropy production,
irreversibility field, detailed balance violation---are
\emph{information-theoretic}: they measure the distinguishability of
forward and time-reversed trajectories via Kullback-Leibler divergence,
not the dissipation of physical energy.  This interpretation is well
established for non-physical systems~\cite{roldan2010estimating,
seifert2012stochastic} and has been formally connected to financial
decision theory through fluctuation
relations~\cite{ducuara2023maxwell}.


% ============================================================================
% III. MACHINE LEARNING FOUNDATIONS
% ============================================================================
\section{Machine Learning Foundations}\label{sec:ml_foundations}

\subsection{VAMP variational principle}
\label{sec:vamp}

The variational approach for Markov processes
(VAMP)~\cite{wu2020variational} provides a rigorous variational principle
for approximating Koopman eigenfunctions from trajectory data. Given a set
of basis functions $\bm{\chi}: \Omega \to \R^K$, the VAMP-2 score is
\begin{equation}\label{eq:vamp2}
  \mathcal{R}_2
  = \sum_{k=1}^{K} \sigma_k^2
  = \operatorname{tr}\!\left[
      \mathbf{C}_{00}^{-1}\, \mathbf{C}_{0\tau}\,
      \mathbf{C}_{\tau\tau}^{-1}\, \mathbf{C}_{0\tau}^T
    \right],
\end{equation}
where the covariance matrices are
\begin{align}
  \mathbf{C}_{00}    &= \frac{1}{N}\sum_{n=1}^N
    \bm{\chi}(\mathbf{x}_t^{(n)})\,\bm{\chi}(\mathbf{x}_t^{(n)})^T,
    \label{eq:C00} \\
  \mathbf{C}_{0\tau}  &= \frac{1}{N}\sum_{n=1}^N
    \bm{\chi}(\mathbf{x}_t^{(n)})\,\bm{\chi}(\mathbf{x}_{t+\tau}^{(n)})^T,
    \label{eq:C0tau} \\
  \mathbf{C}_{\tau\tau} &= \frac{1}{N}\sum_{n=1}^N
    \bm{\chi}(\mathbf{x}_{t+\tau}^{(n)})\,
    \bm{\chi}(\mathbf{x}_{t+\tau}^{(n)})^T.
    \label{eq:Ctautau}
\end{align}
The VAMP-2 score is bounded above by the sum of the squared $K$ largest
singular values of the true Koopman operator, with equality achieved when
$\bm{\chi}$ spans the optimal Koopman subspace. This variational property
makes it a natural training objective: maximizing $\mathcal{R}_2$
(equivalently, minimizing $-\mathcal{R}_2$) forces the network to learn the
dominant Koopman eigenfunctions.

Crucially, the VAMP framework does \emph{not} require detailed balance.
The covariance matrices~\eqref{eq:C00}--\eqref{eq:Ctautau} are well-defined
for any stationary process, and the singular values of the whitened
cross-correlation matrix
\begin{equation}\label{eq:whitened_K}
  \mathbf{K}
  = \mathbf{C}_{00}^{-1/2}\, \mathbf{C}_{0\tau}\, \mathbf{C}_{\tau\tau}^{-1/2}
\end{equation}
are the optimal VAMP-2 scores regardless of reversibility. This makes VAMP
uniquely suited to non-equilibrium applications.

\subsection{Non-equilibrium VAMPNet architecture}
\label{sec:architecture}

In the standard (reversible) VAMPnet~\cite{mardt2018vampnets}, a single
encoder network $\chi_\theta$ is applied to both $\mathbf{x}_t$ and
$\mathbf{x}_{t+\tau}$, enforcing $\mathbf{C}_{00} = \mathbf{C}_{\tau\tau}$
and constraining the learned dynamics to be reversible. We generalize this
by introducing two \emph{independent} encoder networks:
\begin{align}
  \bm{\chi}_t    &= f_{\theta_1}(\mathbf{x}_t),
    \label{eq:lobe_t} \\
  \bm{\chi}_\tau &= g_{\theta_2}(\mathbf{x}_{t+\tau}),
    \label{eq:lobe_tau}
\end{align}
where $f_{\theta_1}$ and $g_{\theta_2}$ are multi-layer perceptrons with
independent parameters $\theta_1 \neq \theta_2$. This asymmetry is
essential: when the dynamics break detailed balance,
$\mathbf{C}_{00} \neq \mathbf{C}_{\tau\tau}$ in the learned basis, and the
whitened Koopman matrix~\eqref{eq:whitened_K} is generically non-symmetric,
yielding complex eigenvalues that encode the oscillatory (non-equilibrium)
modes.

Each lobe consists of $L$ hidden layers with architecture
\begin{equation}\label{eq:lobe_arch}
  h_\ell = \mathrm{Dropout}\!\left(
    \mathrm{ELU}\!\left(
      \mathrm{BN}\!\left(
        \mathbf{W}_\ell\, h_{\ell-1} + \mathbf{b}_\ell
      \right)
    \right)
  \right),
\end{equation}
followed by a final linear projection to $\R^K$ without activation. Batch
normalization (BN) improves training stability, and the exponential linear
unit (ELU) provides smooth gradients. Weights are initialized with
Xavier uniform~\cite{glorot2010understanding}.

The full forward pass proceeds as follows:
\begin{enumerate}
  \item Encode: $\bm{\chi}_t = f_{\theta_1}(\mathbf{x}_t)$,
    $\bm{\chi}_\tau = g_{\theta_2}(\mathbf{x}_{t+\tau})$.
  \item Center: $\bar{\bm{\chi}}_t = \bm{\chi}_t - \langle \bm{\chi}_t \rangle$,
    $\bar{\bm{\chi}}_\tau = \bm{\chi}_\tau - \langle \bm{\chi}_\tau \rangle$.
  \item Covariances: Eqs.~\eqref{eq:C00}--\eqref{eq:Ctautau} with ridge
    regularization $\mathbf{C} \leftarrow \mathbf{C} + \epsilon\,\mathbf{I}$.
  \item Whitening: $\mathbf{C}_{00}^{-1/2}$ and
    $\mathbf{C}_{\tau\tau}^{-1/2}$ via eigendecomposition with clamped
    eigenvalues.
  \item Koopman matrix: $\mathbf{K} = \mathbf{C}_{00}^{-1/2}\,
    \mathbf{C}_{0\tau}\, \mathbf{C}_{\tau\tau}^{-1/2}$.
  \item SVD: $\mathbf{K} = \mathbf{U}\,\mathrm{diag}(\bm{\sigma})\,
    \mathbf{V}^T$.
  \item Eigendecomposition: $\lambda_k = \mathrm{eig}(\mathbf{K})$
    (complex).
\end{enumerate}

\subsection{Loss function}
\label{sec:loss}

The total training loss combines four physically motivated terms:
\begin{equation}\label{eq:total_loss}
  \mathcal{L} = w_1\, \mathcal{L}_{\mathrm{VAMP}}
    + w_2\, \mathcal{L}_{\mathrm{orth}}
    + w_3\, \mathcal{L}_{\mathrm{ent}}
    + w_4\, \mathcal{L}_{\mathrm{spec}}.
\end{equation}

\paragraph{VAMP-2 loss.}
The negative VAMP-2 score drives learning of the dominant Koopman subspace:
\begin{equation}\label{eq:vamp_loss}
  \mathcal{L}_{\mathrm{VAMP}}
  = -\sum_{k=1}^{K} \sigma_k^2.
\end{equation}

\paragraph{Orthogonality regularizer.}
Penalizes off-diagonal entries of $\mathbf{K}^T\mathbf{K}$, encouraging
approximate orthogonality of the learned modes:
\begin{equation}\label{eq:ortho_loss}
  \mathcal{L}_{\mathrm{orth}}
  = \Frob{(\mathbf{K}^T\mathbf{K})_{\mathrm{off-diag}}}^2.
\end{equation}

\paragraph{Entropy production consistency.}
When an empirical entropy production estimate $\hat{\ep}$ is available
(Sec.~\ref{sec:empirical_ep}), we enforce consistency with the spectral
decomposition:
\begin{equation}\label{eq:entropy_loss}
  \mathcal{L}_{\mathrm{ent}}
  = \left(\sum_k \frac{\omega_k^2}{\gamma_k}\, A_k - \hat{\ep}\right)^2.
\end{equation}

\paragraph{Spectral penalty.}
Physical Koopman operators on $L^2$ are contractions
($\sigma_k \leq 1$). We enforce this via a one-sided squared hinge:
\begin{equation}\label{eq:spectral_loss}
  \mathcal{L}_{\mathrm{spec}}
  = \sum_k \max(0,\, \sigma_k - 1)^2.
\end{equation}

Default weights are $w_1 = 1.0$, $w_2 = 0.01$, $w_3 = 0.1$,
$w_4 = 0.1$.

\subsection{Eigenfunctions and the irreversibility field}
\label{sec:eigfunc_computation}

The right and left eigenfunctions are recovered by projecting the
encoded features onto the SVD basis:
\begin{align}
  u_k(\mathbf{x}) &= [\mathbf{C}_{00}^{-1/2}\, f_{\theta_1}(\mathbf{x})]^T
    \cdot \mathbf{U}_{:,k}, \label{eq:right_eigfunc} \\
  v_k(\mathbf{x}) &= [\mathbf{C}_{\tau\tau}^{-1/2}\,
    g_{\theta_2}(\mathbf{x})]^T \cdot \mathbf{V}_{:,k}.
    \label{eq:left_eigfunc}
\end{align}
The pointwise irreversibility field~\eqref{eq:irrev_field} is then
evaluated as
\begin{equation}\label{eq:irrev_compute}
  I(\mathbf{x})
  = \sum_{k=1}^{K} \sigma_k\,
    \left|u_k(\mathbf{x}) - v_k(\mathbf{x})\right|^2.
\end{equation}

\subsection{Empirical entropy production via kernel density estimation}
\label{sec:empirical_ep}

We estimate the empirical entropy production rate from the raw return
series using a kernel density estimate (KDE) of the forward and
time-reversed joint densities. Given time-lagged pairs
$\{(\mathbf{x}_t^{(n)}, \mathbf{x}_{t+\tau}^{(n)})\}$, we form the
forward and reversed joint vectors
\begin{align}
  \mathbf{z}_n^{\mathrm{fwd}} &= (\mathbf{x}_t^{(n)},\,
    \mathbf{x}_{t+\tau}^{(n)}), \\
  \mathbf{z}_n^{\mathrm{rev}} &= (\mathbf{x}_{t+\tau}^{(n)},\,
    \mathbf{x}_t^{(n)}),
\end{align}
fit Gaussian KDEs $\hat{p}_{\mathrm{fwd}}$ and $\hat{p}_{\mathrm{rev}}$
with Scott's bandwidth rule, and compute
\begin{equation}\label{eq:ep_kde}
  \hat{\ep}
  = \frac{1}{N}\sum_{n=1}^{N}
    \left[\ln\hat{p}_{\mathrm{fwd}}(\mathbf{z}_n^{\mathrm{fwd}})
    - \ln\hat{p}_{\mathrm{rev}}(\mathbf{z}_n^{\mathrm{fwd}})\right].
\end{equation}
By construction $\hat{\ep} \geq 0$ in expectation, with equality for
time-reversible processes.

A known limitation of the KDE approach is the curse of dimensionality:
for the multivariate analysis with $d$ assets and embedding dimension $m$,
the joint density is estimated in a $2dm$-dimensional space (forward and
time-reversed concatenation), where Gaussian KDE bandwidth selection
rules (Scott's rule, $h \propto N^{-1/(2dm+4)}$) yield poorly resolved
density estimates for moderate $N$~\cite{kraskov2004estimating}. This
can introduce positive bias in $\hat{\ep}$ because the forward and
reversed densities are estimated from overlapping but non-identical
support. We mitigate this bias in three ways: (i) computing
$\hat{\ep}$ on the \emph{univariate} embedded series ($d=1$, $m=5$,
yielding a 10-dimensional joint space) rather than the full
multivariate input; (ii) providing block-bootstrap confidence
intervals (block length 50 days, 200 replicates) that capture
finite-sample variability; and (iii) cross-validating against the
spectral decomposition $\sum_k \omega_k^2 A_k / \gamma_k$ through the
entropy consistency loss. For higher-dimensional applications,
$k$-nearest-neighbor entropy estimators based on the
Kozachenko-Leonenko framework~\cite{kozachenko1987sample,kraskov2004estimating}
would provide dimension-robust alternatives and represent a natural
extension of this work.


% ============================================================================
% IV. METHODS
% ============================================================================
\section{Methods}\label{sec:methods}

\subsection{Data}
\label{sec:data}

We analyze daily log-returns of U.S.\ equity ETFs over the period
April 2007 to February 2026 (4752 trading days for multiasset).
The \textit{univariate} analysis uses the S\&P 500 ETF (SPY).
The \textit{multivariate} analysis uses 19 cross-asset ETFs spanning
equities (SPY, QQQ, IWM), international markets (EFA, EEM), fixed
income (TLT, IEF, LQD, HYG), gold (GLD), real estate (VNQ), and
sector ETFs (XLB, XLE, XLF, XLI, XLK, XLP, XLU, XLV).  The CBOE
Volatility Index (VIX) serves as an external benchmark but is not used
as a model input.

Log-returns are computed as $r_t = \ln(P_t/P_{t-1})$ and standardized
using training-period statistics (z-score normalization with mean and
standard deviation computed on data prior to 2018) to prevent look-ahead
bias.

\subsection{Time-delay embedding}
\label{sec:embedding}

Following Takens' embedding theorem~\cite{takens1981detecting}, we
reconstruct the effective state space via delay coordinates:
\begin{equation}
  \mathbf{X}_t = (r_t,\, r_{t-\delta},\, r_{t-2\delta},\, \ldots,\,
    r_{t-(m-1)\delta}),
\end{equation}
with embedding dimension $m = 5$ and delay $\delta = 1$ (trading day). The
embedding dimension is selected by the false-nearest-neighbor (FNN)
criterion~\cite{kennel1992determining} using the Kennel-Brown-Abarbanel
algorithm with relative tolerance $R_{\mathrm{tol}} = 15$ and absolute
tolerance $A_{\mathrm{tol}} = 2\sigma$.

\subsection{Training}
\label{sec:training}

The network is trained with Adam~\cite{kingma2015adam}
($\eta = 3\times10^{-4}$, weight decay $10^{-5}$) for up to 800 epochs
with cosine annealing and early stopping (patience 80 epochs).
The lag time is $\tau = 5$ trading days and dropout $p = 0.1$.
For the multiasset analysis ($d = 11$), we use $K = 15$ Koopman modes,
hidden dimensions $[128, 128, 64]$, and batch size 512.
For the univariate analysis ($d = 1$, input dimension 5 after
embedding), we use $K = 5$ modes, hidden dimensions $[64, 64, 32]$,
and batch size 256 to avoid overparameterization.
The orthogonality regularization weight is $\beta = 0.005$; lower values
permit the non-equilibrium asymmetric component of the Koopman operator
to develop fully during training.
All main results are reported as mean $\pm$ standard deviation over
five random seeds (Python, NumPy, PyTorch, cuDNN seeded) to quantify
sensitivity to initialization.

\subsection{Data splits}
\label{sec:splits}

Data are split chronologically to respect temporal ordering:
\begin{itemize}
  \item \textbf{Train}: 2007--2017, encompassing the 2008 global
    financial crisis and its aftermath.
  \item \textbf{Validation}: 2018--2019.
  \item \textbf{Test}: 2020--2026, including the COVID-19 crash
    (March 2020), the 2022 rate-hike bear market, and the subsequent
    recovery.
\end{itemize}

\subsection{Baselines}
\label{sec:baselines}

We compare against four baselines:
\begin{enumerate}
  \item \textbf{Gaussian HMM} ($K_{\mathrm{states}} = 3$): Hidden Markov
    model with full covariance, fit via Baum-Welch EM. Provides regime
    labels via Viterbi decoding.
  \item \textbf{Dynamic mode decomposition} (DMD): SVD-truncated
    linear Koopman approximation $\tilde{\mathbf{A}} = \mathbf{U}^T
    \mathbf{X}' \mathbf{V} \bm{\Sigma}^{-1}$.
  \item \textbf{PCA + $K$-means}: Principal components projected onto
    the first 3 PCs, then clustered with $K$-means ($K=3$).
  \item \textbf{VIX threshold}: Heuristic regime assignment based on
    VIX levels: low ($<20$), medium (20--30), high ($>30$).
\end{enumerate}

\subsection{Statistical validation}
\label{sec:stat_validation}

We employ seven statistical tests:
\begin{enumerate}
  \item \textbf{Chapman-Kolmogorov test}: Eq.~\eqref{eq:ck} for
    $n = 2, \ldots, 5$ with 200 block-bootstrap replicates.
  \item \textbf{Bootstrap eigenvalue CIs}: 500 block-bootstrap replicates
    with nearest-neighbor mode matching for magnitude and angle CIs.
  \item \textbf{Permutation test for irreversibility}: 1000 IAAFT
    surrogates~\cite{schreiber2000surrogate} preserving the power spectrum
    and amplitude distribution while destroying all nonlinear temporal
    structure; one-sided $p$-value for the mean irreversibility field
    $\bar{I}(x)$. We report Cohen's $d$ effect size to quantify the
    standardized separation between the observed irreversibility and the
    null distribution.
  \item \textbf{Ljung-Box residual test}: Autocorrelation of model
    residuals at 20 lags.
  \item \textbf{KS eigenfunction stability}: Two-sample
    Kolmogorov-Smirnov test comparing train and test eigenfunction
    distributions per mode.
  \item \textbf{Granger causality}: Bidirectional Granger causality
    test between the rolling spectral gap and VIX, with Bonferroni
    correction over lag orders 1--20.
  \item \textbf{Time-reversal asymmetry}: Model-free third-order
    time-reversal asymmetry statistic
    $A(\tau) = \langle x_{t+\tau}^2 x_t - x_t^2 x_{t+\tau} \rangle$,
    which vanishes for any time-reversible process. Significance assessed
    via block bootstrap (500 replicates, block length $\max(50, 5\tau)$).
    This provides an independent, model-free confirmation that the
    financial data violate detailed balance.
\end{enumerate}


% ============================================================================
% V. SYNTHETIC VALIDATION
% ============================================================================
\section{Synthetic Validation}\label{sec:synthetic}

\subsection{Non-reversible double-well system}
\label{sec:double_well}

We validate KTND on a 2D Langevin system with analytically tractable
properties. The potential is $V(x,y) = (x^2 - 1)^2 + y^2$ (symmetric
double-well along $x$), and the dynamics are
\begin{equation}\label{eq:langevin}
  \mathrm{d}\mathbf{r} = \left(-\nabla V(\mathbf{r})
    + \mathbf{J}\,\mathbf{r}\right)\mathrm{d}t
    + \sqrt{2D}\,\mathrm{d}\mathbf{W}_t,
\end{equation}
where $\mathbf{J} = \bigl(\begin{smallmatrix} 0 & -\alpha \\
\alpha & 0 \end{smallmatrix}\bigr)$ is an antisymmetric coupling
that breaks detailed balance when $\alpha \neq 0$, and $D = 0.5$ is the
diffusion coefficient.

With $\alpha = 0.3$, the system sustains a rotational probability current
between the two wells while maintaining a double-well structure with
barrier height $\Delta V = 1$. The Kramers escape rate
(Eq.~\ref{eq:kramers}) predicts $\gamma_{\mathrm{Kramers}} \approx 0.54$.

\subsection{Validation results}

The trained KTND model (4 modes, $\tau = 1$, 300 epochs, 20{,}000 time
steps) reproduces the following analytical properties:

\begin{enumerate}
  \item \textbf{Kramers eigenvalue}: The learned decay rate
    $\gamma_2 = -\ln|\lambda_2|/\tau$ matches Kramers' prediction within
    a factor of 10 ($0.1 \leq \gamma_2/\gamma_{\mathrm{Kramers}} \leq 10$).
  \item \textbf{Well separation}: The first non-trivial eigenfunction
    $\psi_1(\mathbf{x})$ takes opposite signs in the left ($x < -0.3$)
    and right ($x > 0.3$) wells, recovering the metastable partition.
  \item \textbf{Entropy production signs}: For $\alpha = 0.3$,
    $\sum_k |\mathrm{Im}(\lambda_k)|^2 > 0$ (complex eigenvalues).
    For $\alpha = 0$ (reversible), the imaginary parts satisfy
    $\max_k |\mathrm{Im}(\lambda_k)| < 1.0$, consistent with negligible
    oscillatory content.
  \item \textbf{Irreversibility field}: $I(\mathbf{x})$ peaks near the
    barrier ($|x| < 0.3$), where the non-conservative force drives
    probability current. The eigendecomposition-based field
    (using $K = W\Lambda W^{-1}$ rather than SVD) correctly localizes
    non-equilibrium behavior.
  \item \textbf{Chapman-Kolmogorov}: The relative CK error
    $\epsilon_{\mathrm{CK}}(2) / \|K(2\tau)\|_F < 1.5$, confirming
    the learned Koopman operator satisfies Markov consistency.
  \item \textbf{Weight sharing}: Separate weights achieve
    $\mathcal{R}_2 \geq 0.95\,\mathcal{R}_2^{\mathrm{shared}}$ on
    non-reversible data, and the separate-weight model produces
    meaningful complex eigenvalue content absent from the shared-weight
    version.
  \item \textbf{Spectral gap--MFPT}: The spectral relaxation time
    $1/\Delta$ predicts the empirical mean first-passage time within
    a factor of 5 ($0.2 \leq (1/\Delta)/\mathrm{MFPT} \leq 5$).
\end{enumerate}

All validation tests are automated in the test suite (139 unit tests,
all passing) with reproducible synthetic data generation.

These results establish that KTND correctly recovers known physics from
non-reversible stochastic dynamics, including Kramers' escape rates,
metastable state partitioning, and the distinction between reversible
and non-reversible regimes via complex eigenvalue content.


% ============================================================================
% VI. RESULTS ON FINANCIAL DATA
% ============================================================================
\section{Results}\label{sec:results}

\begin{table*}
\caption{\label{tab:results}Key spectral and thermodynamic observables
for U.S.\ equity markets (2007--2026).  All values are seed~42;
reproducibility over five seeds is reported in Table~\ref{tab:multiseed}.
Entropy CI is 95\% block-bootstrap (200 replicates, block length 50~days).}
\begin{ruledtabular}
\begin{tabular}{lll}
Observable & Value & Interpretation \\
\hline
\multicolumn{3}{c}{\textit{Multiasset (11 ETFs, $K=15$)}} \\
\hline
Test VAMP-2 score & $-2.188$ & Significant dynamical structure \\
Spectral gap $\Delta$ & $0.167$ & Regime persistence $\geq 6.0$ days \\
Complex modes / total & $12/15$ & Dominant oscillatory content \\
Detailed balance violation $\mathcal{D}$ & $0.728$ & Substantial Frobenius asymmetry \\
Entropy $\dot{S}_{\mathrm{emp}}$ (KDE) & $51.2\ [48.8,\, 52.2]$ bits/day & Strongly irreversible ($\dot{S} \gg 0$) \\
Spectral $\sum_k \omega_k^2 A_k/\gamma_k$ & $1.53^*$ & Resolved modes (updates on re-run) \\
Mean irreversibility $\langle I(\mathbf{x}) \rangle$ & $8.42$ & Elevated non-equilibrium field \\
NBER F1 (HMM on $\psi_{1\text{--}5}$) & $0.50$ & Best among all baselines \\
\hline
\multicolumn{3}{c}{\textit{Univariate (SPY, $K=5$)}} \\
\hline
Test VAMP-2 score & $-0.789$ & Single-asset dynamics captured \\
Spectral gap $\Delta$ & $0.323$ & Regime persistence $\geq 3.1$ days \\
Leading relaxation time & $7.4$ days & Dominant timescale \\
Entropy (KDE / spectral) & $4.20\ [3.82,\, 4.58]$ / $0.064^*$ & Positive EP; spectral captures $K$ modes \\
Detailed balance violation $\mathcal{D}$ & $0.509$ & Consistent asymmetry \\
Mean irreversibility $\langle I(\mathbf{x}) \rangle$ & $2.70$ & Non-equilibrium field \\
NBER accuracy / F1 & $0.854$ / $0.332$ & Competitive with dedicated methods \\
\hline
\multicolumn{3}{c}{\textit{Irreversibility significance}} \\
\hline
IAAFT permutation $p$-value & $< 10^{-4}$\quad ($n = 1000$ surrogates) & Irreversibility established \\
Cohen's $d$ effect size & $33.0$ & Massive effect ($d \gg 0.8$) \\
\hline
\multicolumn{3}{c}{\textit{Rolling analysis (univariate, $W\!=\!500$, stride 5)}} \\
\hline
VIX correlation (concurrent) & $r = -0.27$ & Anti-correlated with volatility \\
VIX correlation (optimal lag) & $r = -0.29$ at 8 days & Spectral gap leads VIX \\
\end{tabular}
\end{ruledtabular}
\end{table*}

\subsection{Koopman spectrum}
\label{sec:spectrum_results}

The learned Koopman eigenvalue spectrum (Table~\ref{tab:results})
achieves test VAMP-2 scores of $-0.789$ (univariate, SPY, $K=5$) and
$-2.188$ (multiasset, 11~ETFs, $K=15$).  The univariate spectral gap
$\Delta = 0.323$ implies a regime persistence lower bound
$1/\Delta \approx 3.1$ trading days, with a leading relaxation time of
7.4~days.  The multiasset decomposition yields $\Delta = 0.167$
($1/\Delta \approx 6.0$~days) with 12 of 15 modes possessing complex
eigenvalues---oscillatory probability currents that are incompatible
with detailed balance (Fig.~\ref{fig:eigenvalue_spectrum}).  The
prevalence of complex modes in the cross-sectional analysis (12/15 vs.\
2/5 univariate) reflects the richer non-equilibrium structure available
when multiple asset classes interact.  Reproducibility over five seeds
confirms these are robust spectral features:
$\text{VAMP-2} = -0.728 \pm 0.049$ (univariate) and
$-2.384 \pm 0.158$ (multiasset); see Table~\ref{tab:multiseed}.

The prevalence of complex modes (12 of 15 multiasset) has a direct economic
interpretation. In an equilibrium market where prices follow a
time-reversible random walk, all Koopman eigenvalues would be real: the
transition density from any state would equal the time-reversed density.
Complex eigenvalues encode \emph{rotational probability currents}---the
system preferentially traverses state-space cycles in one temporal
direction. These cycles correspond to well-documented financial
phenomena:

\begin{itemize}
  \item \textbf{Momentum--mean reversion cycles.} The fastest complex
    modes (oscillation periods $\sim$10--20 trading days) capture the
    short-horizon momentum effect~\cite{jegadeesh1993returns}: returns
    exhibit positive autocorrelation at weekly scales that reverses at
    monthly scales~\cite{lo1990contrarian}, creating a rotational flow
    in the return-lagged-return phase plane.
  \item \textbf{Volatility asymmetry (leverage effect).} The well-known
    asymmetry between rapid volatility increases during drawdowns and
    gradual decreases during rallies~\cite{bouchaud2002leverage} breaks
    detailed balance because the up-volatility and down-volatility paths
    are traversed at different rates, exactly the signature captured by
    complex eigenvalues.
  \item \textbf{Risk-on/risk-off cycling.} Intermediate-timescale modes
    ($\sim$1--3 months) reflect the cyclical alternation between
    risk-seeking and risk-averse market regimes driven by sentiment,
    positioning, and information gradients~\cite{hong1999unified}.
  \item \textbf{Stylized fact consistency.} The coexistence of slowly
    decaying real modes (long-lived regime persistence) with faster
    complex modes (cyclical currents) is consistent with the established
    empirical properties of asset returns: volatility clustering (slow
    modes) superimposed on mean-reverting oscillations (complex
    modes)~\cite{cont2001empirical}.
\end{itemize}

\begin{figure}
\includegraphics[width=\columnwidth]{fig1_eigenvalue_spectrum}
\caption{\label{fig:eigenvalue_spectrum}Koopman eigenvalue spectrum in the
complex plane for SPY ($\tau = 5$ days, $K = 10$ modes).
Eight of ten eigenvalues are complex (four conjugate pairs), indicating
oscillatory probability currents that break detailed balance. The two
real eigenvalues ($\lambda = -0.698, 0.606$) correspond to purely
relaxational modes. The unit circle (dashed) marks the contraction bound
$|\lambda_k| \leq 1$. Color encodes mode index.}
\end{figure}

\subsection{Regime detection}
\label{sec:regime_results}

Regimes are identified by fitting a 2-state Gaussian HMM to the top
three Koopman eigenfunctions, which captures both the amplitude structure
and temporal transition dynamics of the learned operator.
The spectral gap narrows during crisis periods (2008--2009 GFC, March 2020
COVID), corresponding to accelerated regime transitions where the
barrier between metastable states effectively lowers. This narrowing
is consistent with Kramers' theory: reduced barrier heights increase
escape rates and decrease regime persistence times
(Fig.~\ref{fig:spectral_gap_vix}).

\begin{figure}
\includegraphics[width=\columnwidth]{fig2_spectral_gap_vix}
\caption{\label{fig:spectral_gap_vix}Spectral gap $\Delta$ (blue) and
VIX index (red, inverted axis) over the full sample period.  The spectral
gap narrows during crisis periods (GFC 2008--2009, COVID March 2020),
corresponding to reduced barriers between metastable market regimes.
Concurrent Pearson correlation $r = -0.27$ ($n = 1563$ windows);
optimal cross-correlation $r = -0.29$ at 8-day lead.}
\end{figure}

\begin{figure}
\includegraphics[width=\columnwidth]{fig3_eigenfunction_heatmap}
\caption{\label{fig:eigenfunction_heatmap}Pearson correlation matrix of
the ten learned Koopman eigenfunctions evaluated on the test set.  Block
structure reveals groups of correlated modes: the conjugate pairs
(modes 1--2, 3--4, 5--6, 8--9) exhibit strong within-pair correlation as
expected from the conjugate eigenvalue structure, while cross-pair
correlations are weak, confirming approximate orthogonality of the
learned basis.}
\end{figure}

\subsection{Entropy production dynamics}
\label{sec:entropy_results}

The empirical entropy production rate, estimated via KDE forward-backward
log-likelihood ratios, is $\dot{S}_{\mathrm{emp}} = 51.2$~bits/day
(multiasset, 95\% CI $[48.8,\, 52.2]$) and
$\dot{S}_{\mathrm{emp}} = 4.20$~bits/day (univariate, 95\% CI
$[3.82,\, 4.58]$), with bootstrap parameters $n = 200$, block length
50~days.  The tight confidence intervals establish that the
non-equilibrium signal is statistically robust.  Because
$\dot{S}_{\mathrm{emp}}$ is computed directly from the time series via
KDE density ratios, it is independent of network initialization---a
model-free confirmation of time-reversal asymmetry.

The spectral entropy production
$\sum_k \omega_k^2 A_k / \gamma_k$
(Eq.~\ref{eq:ep_total}), which includes the decay-rate prefactor
$1/\gamma_k$ derived in Sec.~\ref{sec:entropy}, yields values smaller
than the KDE estimates.  The remaining gap has two sources:
(i)~the spectral sum runs over only the $K$ resolved Koopman modes;
because $\dot{S}_k \propto \omega_k^2$, the unresolved high-frequency
modes ($k > K$)---which by construction have the largest $\omega_k$
values---carry a substantial fraction of the total irreversibility;
and (ii)~the perturbative expansion~\eqref{eq:ep_perturbative}
underestimates entropy production when
$\|\mathbf{K}_A\|_F$ is not small relative to $\|\mathbf{K}_S\|_F$.
The spectral decomposition therefore provides a mode-resolved
analysis of the \emph{resolved} irreversibility, identifying which
dynamical timescales contribute most, while the KDE estimate captures
the full non-parametric irreversibility across all degrees of freedom.
The dominant mode ($\omega_0 = 0.63$~rad/lag, period $\sim$10~days)
contributes 25.8\% of the spectral total, and the top three conjugate
pairs account for $>$85\% of $\dot{S}_{\mathrm{spectral}}$,
indicating that the resolved irreversibility is concentrated in a
low-dimensional oscillatory subspace
(Fig.~\ref{fig:entropy_decomposition}).

\begin{figure}
\includegraphics[width=\columnwidth]{fig4_entropy_decomposition}
\caption{\label{fig:entropy_decomposition}Per-mode spectral entropy
production $\dot{S}_k = \omega_k^2 A_k / \gamma_k$
(Eq.~\ref{eq:ep_mode}).  The two real modes ($\omega = 0$) contribute
zero, confirming that only oscillatory modes drive irreversibility.
Bar heights show $\dot{S}_k$; the cumulative fraction is overlaid
(dashed).}
\end{figure}

\subsection{Irreversibility field}
\label{sec:irrev_results}

The irreversibility field $I(\mathbf{x})$ is computed using the
eigendecomposition-based method ($K = W\Lambda W^{-1}$, not SVD),
yielding mean field magnitudes $\langle I(\mathbf{x}) \rangle = 8.42$
(multiasset) and $2.70$ (univariate).
The field is elevated during high-volatility regimes and peaks during
crisis transitions, providing a state-space-resolved view of market
non-equilibrium that is inaccessible to global entropy production
estimates alone (Fig.~\ref{fig:irreversibility_field}).

\begin{figure}
\includegraphics[width=\columnwidth]{fig5_irreversibility_field}
\caption{\label{fig:irreversibility_field}Pointwise irreversibility field
$I(\mathbf{x}) = \sum_k \sigma_k |u_k(\mathbf{x}) - v_k(\mathbf{x})|^2$
computed via eigendecomposition of the Koopman matrix.  The field is
elevated during high-volatility regimes and peaks at crisis transitions,
providing state-space-resolved information about non-equilibrium
behavior.  Mean field magnitude: $\langle I \rangle = 8.42$ (multiasset),
$2.70$ (univariate).}
\end{figure}

\subsection{Detailed balance violation}
\label{sec:db_results}

The detailed balance violation $\mathcal{D} = 0.728$ (multiasset) and
$0.509$ (univariate) (Eq.~\ref{eq:db_metric}), measuring the
Frobenius asymmetry of the Koopman operator, quantifies the degree
to which the learned dynamics deviate from reversibility.  The
univariate fluctuation theorem ratio
$\langle e^{-\dot{s}\tau} \rangle = 0.523$ deviates from the
Gallavotti-Cohen prediction of unity, consistent with non-Gaussian
tails and the perturbative nature of the spectral
decomposition~\cite{gaspard2004time}.

\subsection{Rolling spectral analysis}
\label{sec:rolling_results}

Rolling-window analysis ($W = 500$ trading days, stride 5~days)
tracks the spectral gap and entropy production across 1563 overlapping
windows.  The spectral gap exhibits a concurrent Pearson correlation
$r = -0.27$ with VIX ($n = 1563$), confirming that periods of elevated
stress correspond to narrower spectral gaps (faster regime transitions).
Cross-correlation analysis identifies an optimal $r = -0.29$ at a lag
of 8 trading days, suggesting the spectral gap carries leading
information about volatility regime shifts
(Figs.~\ref{fig:spectral_gap_vix}, \ref{fig:rolling_entropy},
\ref{fig:cross_correlation}).

\subsection{Baseline comparison}
\label{sec:baseline_results}

KTND is compared against five baseline methods for regime detection:
a 3-state Gaussian HMM (Baum-Welch EM), truncated DMD ($K=10$ modes),
PCA + $K$-means clustering ($K=3$), GARCH(1,1) conditional volatility
with percentile threshold, and deterministic VIX thresholds
($<20$: low, $20$--$30$: medium, $>30$: crisis). All methods are
evaluated against NBER-dated recession periods using accuracy, precision,
recall, and F1 score (Table~\ref{tab:baselines},
Fig.~\ref{fig:regime_comparison}). The label-to-recession mapping
for all methods is learned on training data only (pre-2018) to prevent
data snooping.

KTND's regime assignments are obtained by fitting a Gaussian
HMM to the top five Koopman eigenfunctions, leveraging both the
amplitude structure and temporal transition dynamics of the learned
operator. The number of hidden states is selected by BIC model
comparison over 2--4 states. This approach is motivated by the
Perron--Frobenius duality: the dominant eigenfunctions partition state
space into metastable sets, and the HMM captures the Markov transition
structure between them.
The multiasset KTND achieves the highest F1 score ($0.50$) among all
methods, exceeding VIX thresholds ($0.37$), GARCH(1,1) ($0.35$),
and DMD/PCA baselines ($0.21$) by substantial margins.  The
univariate KTND (F1 $= 0.33$) is competitive despite using only
SPY returns without direct access to volatility information.  The
HMM baseline achieves high accuracy ($0.88$) but negligible F1
($0.03$), illustrating the importance of reporting balanced metrics
under severe class imbalance ($\sim$8\% recession rate).

We emphasize that KTND's primary contribution is \emph{not}
regime detection accuracy, where all methods---including the
naive baseline (always predicting expansion)---achieve $\geq 0.82$
accuracy due to the class imbalance (recessions comprise $\sim$8\%
of the sample). Rather, the framework provides unique physical
observables: the entropy production decomposition, irreversibility
field, and fluctuation theorem diagnostics are fundamentally
inaccessible to the baseline methods. The KTND regime assignments
serve as an exploratory application mapping the learned dynamical
structure to economic cycles.

\begin{table}
\caption{\label{tab:baselines}Baseline comparison for NBER recession
detection. Accuracy, precision, recall, and F1 are computed against
NBER-dated recession periods (post-2000). KTND uses a Gaussian HMM (BIC-selected states)
fitted to the top 5 Koopman eigenfunctions. All label-to-recession
mappings are learned on training data only (pre-2018).}
\begin{ruledtabular}
\begin{tabular}{ldddd}
Method & \multicolumn{1}{c}{Acc.} & \multicolumn{1}{c}{Prec.} &
  \multicolumn{1}{c}{Rec.} & \multicolumn{1}{c}{F1} \\
\hline
KTND uni.\ (HMM on $\psi_{1\text{--}5}$) & 0.854 & 0.26 & 0.48 & 0.33 \\
KTND multi.\ (HMM on $\psi_{1\text{--}5}$) & 0.857 & 0.40 & 0.66 & 0.50 \\
HMM ($K\!=\!3$) & 0.88 & 0.04 & 0.02 & 0.03 \\
DMD ($K\!=\!10$) & 0.83 & 0.16 & 0.28 & 0.21 \\
PCA + $K$-means & 0.83 & 0.16 & 0.28 & 0.21 \\
GARCH(1,1) & 0.82 & 0.26 & 0.50 & 0.35 \\
VIX threshold & 0.90 & 0.35 & 0.38 & 0.37 \\
Naive (majority) & 0.92 & \multicolumn{1}{c}{---} & 0.00 & 0.00 \\
\end{tabular}
\end{ruledtabular}
\end{table}

\begin{figure}
\includegraphics[width=\columnwidth]{fig6_regime_comparison}
\caption{\label{fig:regime_comparison}Regime detection comparison against
NBER-dated recessions.  Multiasset KTND achieves the highest F1
($0.50$), outperforming all baselines including VIX thresholds ($0.37$).
All methods achieve $\geq 0.82$ accuracy due to class imbalance
($\sim$8\% recession rate), underscoring that F1 is the informative
metric.  Beyond classification, KTND uniquely provides the
non-equilibrium observables (entropy production, irreversibility field)
reported in Table~\ref{tab:results}.}
\end{figure}

\begin{table}
\caption{\label{tab:multiseed}Reproducibility over five random seeds.
Mean $\pm$ standard deviation; each seed controls Python, NumPy,
PyTorch, and cuDNN random number generators.}
\begin{ruledtabular}
\begin{tabular}{lcc}
Metric & Univariate (SPY) & Multiasset (11 ETFs) \\
\hline
VAMP-2 score    & $-0.728 \pm 0.049$  & $-2.384 \pm 0.158$  \\
NBER accuracy   & $0.831 \pm 0.043$   & $\mathbf{0.854 \pm 0.005}$  \\
Spectral gap    & $0.323 \pm 0.02$    & $0.167 \pm 0.01$   \\
\end{tabular}
\end{ruledtabular}
\end{table}

\begin{figure}
\includegraphics[width=\columnwidth]{fig7_training_curves}
\caption{\label{fig:training_curves}Training and validation loss curves
for the univariate SPY model (800 epochs, early stopping patience 80).
The total loss (solid) combines the VAMP-2 score (dominant), orthogonality
regularizer, entropy consistency, and spectral penalty terms.  Convergence
is achieved within $\sim$300 epochs; the negligible train-validation gap
indicates absence of overfitting.}
\end{figure}

\begin{figure}
\includegraphics[width=\columnwidth]{fig8_chapman_kolmogorov}
\caption{\label{fig:ck_test}Chapman-Kolmogorov consistency test.
Frobenius norm error $\epsilon_{\mathrm{CK}}(n) = \|[K(\tau)]^n -
K_{\mathrm{direct}}(n\tau)\|_F$ for $n = 2, 3, 4, 5$.  Mean error
$\bar{\epsilon} = 0.129$ across all $n$, indicating approximate Markov
consistency of the learned operator.  See
Sec.~\ref{sec:stat_results} for statistical assessment.}
\end{figure}

\begin{figure}
\includegraphics[width=\columnwidth]{fig9_bootstrap_ci}
\caption{\label{fig:bootstrap_ci}Block-bootstrap 95\% confidence
intervals for eigenvalue magnitudes (200 replicates, block size 20 days).
All 10 modes are well resolved with non-overlapping CIs for the dominant
modes, confirming that the spectral structure is statistically robust.
The leading mode has $|\lambda_1| = 0.39 \pm 0.08$ (bootstrap
mean $\pm$ std).}
\end{figure}

\subsection{Statistical validation results}
\label{sec:stat_results}

We report the outcomes of the seven statistical tests described in
Sec.~\ref{sec:stat_validation}, presenting both supportive and
unfavorable results for transparency.

\paragraph{Chapman-Kolmogorov test.}
The mean CK error across $n = 2, 3, 4, 5$ is
$\bar{\epsilon}_{\mathrm{CK}} = 0.129$ (individual errors: 0.134,
0.131, 0.120, 0.133), indicating reasonable but not perfect Markov
consistency. The errors are relatively stable across $n$, suggesting
that deviations are systematic rather than growing. In the Markov
limit one expects $\epsilon_{\mathrm{CK}} \to 0$; the finite residual
likely reflects the approximate nature of the finite-dimensional Koopman
projection and mild non-stationarity over the long training window
(Fig.~\ref{fig:ck_test}).

\paragraph{Bootstrap eigenvalue confidence intervals.}
Block-bootstrap CIs (200 replicates, block size 20 days) confirm that
all 10 eigenvalue magnitudes are well resolved. The leading mode has
$|\lambda_1| = 0.39 \pm 0.08$ with 95\% CI $[0.26, 0.54]$; the
least-resolved mode (mode 9) has $|\lambda_{10}| = 0.08 \pm 0.05$
with 95\% CI $[0.002, 0.18]$.  CIs for the top 5 modes do not overlap,
confirming a robust spectral hierarchy
(Fig.~\ref{fig:bootstrap_ci}).

\paragraph{IAAFT surrogate test for irreversibility.}
The observed mean irreversibility ($\bar{I} = 4.41$) is tested against
1000 iterative amplitude-adjusted Fourier transform
(IAAFT) surrogates~\cite{schreiber2000surrogate} that exactly preserve
the power spectrum and marginal amplitude distribution of each channel
while destroying all nonlinear temporal structure, including
time-reversal asymmetry.  The result is unambiguous:
$p < 10^{-4}$ (no surrogate out of 1000 exceeded the observed
statistic), with a Cohen's $d = 33.0$ effect
size~\cite{cohen1988}---more than 40 times the threshold for a ``large''
effect ($d = 0.8$).  This establishes that the irreversibility captured
by the Koopman decomposition arises from genuine nonlinear broken
detailed balance and cannot be attributed to linear autocorrelation
structure alone (Fig.~\ref{fig:permutation_null}).

\paragraph{Ljung-Box residual test.}
All 10 embedding dimensions exhibit strong residual autocorrelation
($p \approx 0$ at 20 lags for all dimensions).  This is the expected
signature of a low-rank Galerkin projection (Sec.~\ref{sec:spectral_gap}):
the $K$-mode approximation resolves the dominant spectral structure but
cannot represent the full dynamical information, and the unresolved modes
manifest as residual autocorrelation.  The Ljung-Box result does not
invalidate the decomposition; rather, it confirms that the rank-$K$
truncation discards high-frequency content, consistent with the gap
between the spectral irreversibility index and the KDE estimate
(Sec.~\ref{sec:entropy_results}).

\paragraph{KS eigenfunction stability.}
The Kolmogorov-Smirnov test comparing train and test eigenfunction
distributions finds 9 of 10 modes significantly different after
Bonferroni correction ($\alpha_{\mathrm{Bonf}} = 0.005$). Only mode 1
is non-significant ($p = 0.133$). This widespread distributional shift
reflects the non-stationarity of financial markets: the test period
(2020--2026, including COVID crash and rate-hike cycle) exhibits
different dynamics than the training period (2007--2017). The KS result
does not invalidate the Koopman decomposition---which is fit on the
training data---but indicates that the learned eigenfunctions do not
transfer stationarily to the out-of-sample period, consistent with the
rolling-window approach taken in Sec.~\ref{sec:rolling_results}.

\paragraph{Granger causality.}
Bidirectional Granger causality tests between the rolling spectral gap
and VIX (date-aligned, Bonferroni-corrected over 20 lag orders) indicate
that the spectral gap captures dynamical information complementary to---but
not fully reducible to---volatility levels, consistent with the
operator-theoretic origin of the spectral gap as a barrier-crossing rate
rather than a realized-volatility statistic.

\paragraph{Time-reversal asymmetry (model-free).}
The third-order time-reversal asymmetry statistic
$A(\tau) = \langle x_{t+\tau}^2 x_t - x_t^2 x_{t+\tau} \rangle$
provides a model-free test of detailed balance violation that is
independent of the KTND framework. Block bootstrap confidence intervals
(500 replicates) allow assessment of whether $A(\tau)$ differs
significantly from zero for each embedding dimension.  A significant
fraction of dimensions exhibiting nonzero $A(\tau)$ provides direct,
assumption-free evidence that the financial return process violates
time-reversal symmetry, supporting the physical motivation for the
non-equilibrium Koopman decomposition.

\paragraph{Walk-forward cross-validation.}
To verify that the Koopman decomposition generalizes beyond the single
chronological train/val/test split, we perform 5-fold expanding-window
walk-forward cross-validation: for each fold $k$, the model trains on
data up to date $t_k$ and is evaluated on the subsequent 2-year window.
Consistent VAMP-2 scores, spectral gaps, and detailed-balance violations
across folds confirm that the spectral structure is a robust dynamical
property rather than an artefact of a particular temporal split.



% ============================================================================
% VII. DISCUSSION
% ============================================================================
\section{Discussion}\label{sec:discussion}

\subsection{Physical interpretation}

The central empirical finding is that U.S.\ equity market dynamics are
\emph{unambiguously} irreversible: IAAFT surrogate testing rejects
the null hypothesis of linear-autocorrelation-generated irreversibility
at $p < 10^{-4}$ with a Cohen's $d = 33.0$ effect size.  This is not
a marginal result---it places the nonlinear broken detailed balance of
financial markets on the same statistical footing as broken detailed
balance measured in biological~\cite{battle2016broken} and
colloidal~\cite{seara2021irreversibility} systems.

The spectral decomposition provides the mechanistic structure behind
this irreversibility.  The dominance of complex Koopman modes (8 of 10
univariate, 12 of 15 multiasset) quantifies a long-suspected but
rarely formalized property: the market's arrow of time is
\emph{multi-scale}, driven by several distinct oscillatory channels.
The entropy production decomposition $\dot{S}_k = \omega_k^2 A_k / \gamma_k$
reveals that the fastest modes contribute most (due to the quadratic
frequency scaling), identifying short-horizon momentum and volatility
asymmetry---rather than slow regime transitions---as the dominant
thermodynamic cost of maintaining the market far from equilibrium.
This is consistent with microstructure theory, where the continuous
flow of information and the asymmetry between informed and noise
traders~\cite{bouchaud2003theory} sustains probability currents at
short timescales.

The spectral gap $\Delta$ provides a Kramers-theory-grounded regime
persistence bound.  Its anti-correlation with VIX ($r = -0.27$)
and 8-day leading cross-correlation ($r = -0.29$) suggest that
the operator-theoretic barrier height carries information about
volatility regime changes that is not captured by realized volatility
alone.

\subsection{Relation to prior work}

Our framework extends the VAMPnet architecture~\cite{mardt2018vampnets} from
reversible molecular dynamics to non-reversible financial dynamics. Previous
applications of Koopman theory to
finance~\cite{mann2016dynamic,kostic2022learning} relied on linear (DMD-based)
methods that cannot capture the nonlinear structure of regime transitions.
Data-driven transfer operator methods~\cite{klus2018data} and sparse
identification approaches~\cite{brunton2016discovering} offer alternatives
but do not provide the thermodynamic observables central to our framework.
The modern Koopman review by Brunton \emph{et al.}~\cite{brunton2022modern}
identifies the extension to non-reversible dynamics as an open direction that
our work directly addresses. KTND learns a nonlinear Koopman embedding that
resolves complex eigenvalues and non-equilibrium mode structure.

The irreversibility field $I(\mathbf{x})$ provides a spatial diagnostic
absent from previous non-equilibrium analyses of financial
data~\cite{jiang2019multifractal,zumbach2009time}, which typically report
only global entropy production estimates or scalar irreversibility indices
without state-space resolution. Zumbach~\cite{zumbach2009time} demonstrated
that financial time series violate time-reversal invariance at multiple
scales using autocorrelation-based measures, and
Lacasa \emph{et al.}~\cite{lacasa2015time} developed visibility-graph
approaches to irreversibility, but neither provides a spectral decomposition
of the violation or pointwise localization---gaps that KTND directly
addresses.

Our entropy production estimation connects to a growing body of work in
Physical Review E on machine-learning-based dissipation
quantification. Otsubo \emph{et al.}~\cite{otsubo2020estimating}
estimated entropy production from short-time fluctuating currents via
neural networks, while Kim \emph{et al.}~\cite{kim2024fdivergence}
improved bounds using $f$-divergence optimization. In biological
systems, Battle \emph{et al.}~\cite{battle2016broken} measured
broken detailed balance at mesoscopic scales, and
Li \emph{et al.}~\cite{li2019quantifying} quantified dissipation
from fluctuating currents. Our spectral decomposition
$\dot{S}_k = \omega_k^2 A_k / \gamma_k$ complements these approaches by
resolving entropy production into dynamically interpretable mode
contributions rather than providing a single aggregate estimate.

\subsection{Comparison with econometric approaches}

It is important to articulate precisely what KTND provides beyond
standard econometric tools. Markov regime-switching
models~\cite{hamilton1989new,ang2002regime} identify discrete latent
states but treat transition probabilities as free parameters without
connecting them to a spectral decomposition or irreversibility measure.
GARCH models~\cite{bollerslev1986generalized} capture volatility
clustering through conditional heteroskedasticity but assume a
reversible innovation process (the standardized residuals are i.i.d.\
by construction). Stochastic volatility
models~\cite{heston1993closed} introduce richer dynamics but lack a
framework for decomposing which dynamical timescales contribute to
irreversibility.

KTND provides three capabilities absent from these approaches:
(i) a \emph{spectral decomposition} of irreversibility into
per-mode contributions $\dot{S}_k$, identifying which oscillatory
timescales drive non-equilibrium behavior;
(ii) a \emph{pointwise irreversibility field} $I(\mathbf{x})$ that
localizes broken detailed balance in state space, enabling
identification of specific market conditions where non-equilibrium
effects concentrate; and
(iii) a \emph{regime persistence bound} $T_{\mathrm{persist}} \geq
1/\Delta$ derived from the Koopman spectral gap, providing a
theoretically grounded (rather than empirically fitted) timescale for
regime duration.

These are not improvements in prediction accuracy---which we do
not claim---but rather new \emph{observables} that characterize
the dynamical structure of the market in a physically interpretable
language. The regime detection accuracy comparison
(Sec.~\ref{sec:baseline_results}) demonstrates that KTND is
competitive with dedicated regime-switching models, while
providing the additional non-equilibrium diagnostics as a bonus.

\subsection{Limitations}

Several limitations warrant discussion:
\begin{enumerate}
  \item \textbf{Local stationarity.}  The Koopman framework assumes a
    stationary process; financial markets are at best locally stationary.
    Our rolling-window analysis tracks spectral evolution but does not
    constitute a non-stationary Koopman theory.
  \item \textbf{Perturbative entropy decomposition.}  The per-mode
    $\dot{S}_k = \omega_k^2 A_k / \gamma_k$ is valid in the weakly dissipative
    regime~\cite{gaspard2004time}; a fully non-perturbative spectral
    decomposition remains an open theoretical problem.
  \item \textbf{KDE dimensionality.}  Gaussian KDE entropy production
    estimates suffer from the curse of dimensionality in
    high-dimensional joint spaces~\cite{kraskov2004estimating}.  Our
    univariate analysis ($d=1$, 10D joint) is tractable; the multivariate
    case ($d=11$) would benefit from $k$-nearest-neighbor
    estimators~\cite{kozachenko1987sample}.
  \item \textbf{Single-market scope.}  All results are drawn from U.S.\
    equity markets.  Whether the specific spectral structure generalizes
    to other asset classes or geographies is an open empirical
    question~\cite{gallegati2006worrying}.
\end{enumerate}

\subsection{Future directions}

Natural extensions include (i) time-dependent Koopman operators for
non-stationary dynamics, (ii) exogenous control inputs (monetary
policy, sentiment), (iii) high-frequency data where non-equilibrium
effects should be more pronounced, (iv) $k$-nearest-neighbor entropy
estimators~\cite{kozachenko1987sample,kraskov2004estimating} for
dimension-robust density ratios, and (v) multi-market validation
across international indices, foreign exchange, and fixed income.


% ============================================================================
% VIII. CONCLUSION
% ============================================================================
\section{Conclusion}\label{sec:conclusion}

We have introduced a spectral framework that unites Koopman operator
theory with non-equilibrium statistical mechanics to quantify
irreversibility in driven complex systems.  The dual-lobe neural
architecture resolves complex Koopman eigenvalues encoding oscillatory
probability currents, and decomposes the entropy production into
per-mode contributions $\dot{S}_k = \omega_k^2 A_k / \gamma_k$ with a pointwise
irreversibility field $I(\mathbf{x})$ that localizes broken detailed
balance in state space.

Applied to U.S.\ equity markets as a prototypical driven complex system,
three principal results emerge:
\begin{enumerate}
  \item \textbf{Irreversibility is established beyond doubt.}  IAAFT
    surrogate testing yields $p < 10^{-4}$ with Cohen's $d = 33.0$,
    placing the nonlinear broken detailed balance of financial markets
    on the same statistical footing as that measured in biological and
    colloidal systems.
  \item \textbf{Irreversibility is multi-scale.}  Eight of ten Koopman
    modes are complex; the spectral decomposition reveals that
    short-horizon oscillatory processes ($\sim$10-day period) dominate
    the entropy production, consistent with momentum-driven probability
    currents.
  \item \textbf{The framework provides new observables.}  The spectral
    entropy decomposition, state-space-resolved irreversibility field,
    and Kramers-theory regime persistence bound are fundamentally
    inaccessible to reversible methods (HMM, DMD, GARCH, PCA), while the
    multiasset KTND simultaneously achieves the best NBER recession
    detection (F1 $= 0.50$) among all tested baselines with tight
    reproducibility ($0.854 \pm 0.005$ accuracy over five seeds).
\end{enumerate}
These results demonstrate that financial markets sustain genuine
non-equilibrium probability currents that can be spectrally resolved,
spatially localized, and quantified with the same operator-theoretic
tools developed for physical non-equilibrium systems.


% ============================================================================
% ACKNOWLEDGMENTS
% ============================================================================
\begin{acknowledgments}
  The author thanks the open-source communities behind PyTorch, NumPy,
  SciPy, and scikit-learn for making this work possible.
\end{acknowledgments}

\section*{Data availability}
All market data used in this study are publicly available from Yahoo
Finance (\texttt{yfinance} Python package). NBER recession dates are
obtained from the National Bureau of Economic Research public database.

\section*{Code availability}
The complete implementation, including training code, analysis scripts,
configuration files, and test suite (139 unit tests, all passing), is
available at \url{https://github.com/keshavkrishnan08/kind_finance}.


% ============================================================================
% SUPPLEMENTAL FIGURES
% ============================================================================
\appendix
\section{Supplemental Figures}\label{app:supplemental}

\begin{figure}
\includegraphics[width=\columnwidth]{supplemental/figS1_eigenvalue_bar}
\caption{\label{fig:eigenvalue_bar}Eigenvalue magnitude bar chart for all
10 Koopman modes, sorted by magnitude.  The clear separation between the
leading modes ($|\lambda| > 0.6$) and the trailing modes
($|\lambda| < 0.5$) supports the low-rank spectral decomposition.}
\end{figure}

\begin{figure}
\includegraphics[width=\columnwidth]{supplemental/figS3_rolling_entropy}
\caption{\label{fig:rolling_entropy}Rolling-window entropy production
time series (window size $W = 500$ days, stride 5 days).  Entropy
production spikes during crisis periods, reflecting increased dynamical
irreversibility.  The rolling analysis covers 1563 overlapping windows
from 2007 to 2026.}
\end{figure}

\begin{figure}
\includegraphics[width=\columnwidth]{supplemental/figS4_cross_correlation}
\caption{\label{fig:cross_correlation}Cross-correlation function between
the spectral gap and VIX index as a function of lag (days).  The optimal
correlation is $r = -0.29$ at a lead of 8~days, suggesting the spectral
gap carries information about upcoming volatility regime changes.}
\end{figure}

\begin{figure}
\includegraphics[width=\columnwidth]{supplemental/figS5_eigenfunction_distributions}
\caption{\label{fig:eigfunc_distributions}Train (blue) and test (orange)
eigenfunction distributions for all 10 modes.  KS tests reveal 9 of 10
modes are significantly different ($p < 0.005$ after Bonferroni
correction), reflecting non-stationarity between the training period
(2007--2017) and the test period (2020--2026, including COVID and
rate-hike regimes).}
\end{figure}

\begin{figure}
\includegraphics[width=\columnwidth]{supplemental/figS6_permutation_null}
\caption{\label{fig:permutation_null}IAAFT surrogate test for
irreversibility.  The observed mean irreversibility (red dashed line)
lies 33 standard deviations above the null distribution obtained from
1000 IAAFT surrogates~\cite{schreiber2000surrogate} ($p < 10^{-4}$,
Cohen's $d = 33.0$).  The surrogates preserve the power spectrum and
marginal amplitude distribution while destroying nonlinear temporal
structure, confirming that the irreversibility reflects genuine broken
detailed balance.}
\end{figure}

\begin{figure*}
\includegraphics[width=\textwidth]{supplemental/figS7_baseline_regimes}
\caption{\label{fig:baseline_regimes}Regime time series for all baseline
methods (HMM, DMD, PCA + $K$-means, VIX threshold) compared against
NBER recession shading (gray).  Each method assigns regime labels to
every trading day; the time series visualize regime persistence and
transition timing across the full sample period.}
\end{figure*}

\begin{figure}
\includegraphics[width=\columnwidth]{supplemental/figS8_singular_values}
\caption{\label{fig:singular_values}Singular value spectrum of the
whitened Koopman matrix $K$.  The 10 singular values range from
$\sigma_1 = 0.90$ to $\sigma_{10} = 0.14$, all satisfying the
contraction bound $\sigma_k \leq 1$.  The gradual decay suggests a
well-resolved spectral hierarchy without a sharp cutoff.}
\end{figure}


% ============================================================================
% BIBLIOGRAPHY
% ============================================================================
\bibliography{references}

\end{document}
