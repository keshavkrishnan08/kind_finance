% ============================================================================
% Non-Equilibrium Koopman-Thermodynamic Neural Decomposition
% for Financial Market Dynamics
%
% Target: Physical Review E
% ============================================================================

\documentclass[aps,pre,twocolumn,superscriptaddress,showpacs,floatfix]{revtex4-2}

% ---- Packages ----
\usepackage{amsmath,amssymb,amsfonts}
\usepackage{graphicx}
\usepackage{bm}
\usepackage{hyperref}
\usepackage{xcolor}
\usepackage{algorithm}
\usepackage{algpseudocode}
\usepackage{booktabs}
\usepackage{dcolumn}
\usepackage{natbib}
\usepackage{physics}

% ---- Graphics path ----
\graphicspath{{../outputs/figures/}}

% ---- Custom macros ----
\newcommand{\Koop}{\mathcal{K}}
\newcommand{\Trans}{\mathcal{T}}
\newcommand{\Hilb}{\mathcal{H}}
\newcommand{\R}{\mathbb{R}}
\newcommand{\C}{\mathbb{C}}
\newcommand{\E}{\mathbb{E}}
\newcommand{\KL}{\mathrm{KL}}
\newcommand{\ep}{\dot{S}}
\newcommand{\Frob}[1]{\left\lVert #1 \right\rVert_{\mathrm{F}}}

\begin{document}

% ============================================================================
\title{Non-Equilibrium Koopman-Thermodynamic Neural Decomposition\\
for Financial Market Dynamics}

\author{Keshav Krishnan}
\affiliation{Independent Researcher}

\date{\today}

\pacs{89.65.Gh, 05.70.Ln, 02.50.Ga, 05.45.Tp}
% 89.65.Gh -- Economics; econophysics, financial markets, business and management
% 05.70.Ln -- Nonequilibrium and irreversible thermodynamics
% 02.50.Ga -- Markov processes
% 05.45.Tp -- Time series analysis

% ============================================================================
\begin{abstract}
We develop a spectral framework for quantifying irreversibility in driven
complex systems by combining Koopman operator theory with non-equilibrium
statistical mechanics. A dual-lobe variational neural architecture learns
the left and right eigenfunctions of a non-self-adjoint Koopman operator
from time-series data, resolving complex eigenvalues that encode
oscillatory probability currents and broken detailed balance. The
framework decomposes the entropy production rate into per-mode spectral
contributions $\dot{S}_k = \omega_k^2 A_k$ and constructs a pointwise
irreversibility field
$I(\mathbf{x}) = \sum_k \sigma_k |u_k(\mathbf{x}) - v_k(\mathbf{x})|^2$
that localizes departures from equilibrium in state space. We validate
on non-reversible Langevin dynamics, recovering Kramers' escape rates
from the learned spectrum, and apply the method to U.S.\ equity markets
(1993--2025) as a prototypical non-equilibrium complex system driven by
information asymmetry and heterogeneous agent interactions. The spectral
gap yields a regime persistence bound $T_{\mathrm{persist}} \geq 1/\Delta$
that identifies dynamical regimes with characteristic timescales of
$5$--$28$ days. Comparisons with hidden Markov
models, dynamic mode decomposition, GARCH, and PCA baselines---together with
Chapman-Kolmogorov consistency tests, block-bootstrap confidence intervals,
and permutation tests for irreversibility---confirm that the non-equilibrium
spectral decomposition captures dynamical structure inaccessible to
reversible methods. All main results are reported as mean $\pm$ standard
deviation over five random seeds to quantify sensitivity to initialization.
\end{abstract}

\maketitle

% ============================================================================
% I. INTRODUCTION
% ============================================================================
\section{Introduction}\label{sec:intro}

Quantifying irreversibility---the extent to which a system's dynamics
violate time-reversal symmetry---is a central problem in non-equilibrium
statistical mechanics~\cite{seifert2012stochastic,ciliberto2017experiments}.
For physical systems, entropy production provides the fundamental measure of
irreversibility, and recent advances in machine learning have enabled its
estimation from trajectory data in biological~\cite{battle2016broken,
li2019quantifying}, colloidal~\cite{seara2021irreversibility}, and
synthetic~\cite{otsubo2020estimating,kim2024fdivergence} systems.
Financial markets offer a compelling arena for these methods: they are
driven far from equilibrium by information asymmetry, heterogeneous agent
strategies, and regulatory feedback
loops~\cite{bouchaud2003theory,mantegna1999introduction}, producing
well-documented signatures of time-reversal asymmetry including
fat-tailed return distributions, volatility clustering, and the leverage
effect~\cite{cont2001empirical,zumbach2009time}.

Despite this, the equilibrium paradigm has dominated quantitative finance.
The geometric Brownian motion underlying the Black-Scholes
framework~\cite{black1973pricing} and the efficient market
hypothesis~\cite{fama1970efficient} both assume time-reversible price
dynamics---equivalent to detailed balance in the language of statistical
mechanics. Extensions such as GARCH
processes~\cite{bollerslev1986generalized}, stochastic volatility
models~\cite{heston1993closed}, and Markov regime-switching
models~\cite{hamilton1989new,ang2002regime} introduce richer dynamics but
do not provide a systematic framework for quantifying \emph{how far} the
market deviates from equilibrium or for decomposing that deviation into
dynamically interpretable spectral modes.

We note that the application of physics methods to economic systems
has been subject to methodological
critiques~\cite{gallegati2006worrying}, particularly regarding
insufficient engagement with the economics literature, lack of
statistical rigor, and overreliance on physical analogies without formal
justification. We address these concerns explicitly: our framework is
grounded in rigorous operator-theoretic results (not physical analogy),
we benchmark against standard econometric baselines
(HMM~\cite{hamilton1989new}, GARCH~\cite{bollerslev1986generalized}),
and we provide comprehensive statistical validation including
bootstrap confidence intervals, permutation tests, and
Chapman-Kolmogorov consistency checks.

Recent work has begun to forge rigorous connections between
non-equilibrium thermodynamics and finance.
Ducuara \emph{et al.}~\cite{ducuara2023maxwell} proved that Crooks'
fluctuation relations map onto expected utility theory, establishing
a formal mathematical bridge (not merely an analogy) between
stochastic thermodynamics and financial decision theory. Rold{\'a}n
and Parrondo~\cite{roldan2010estimating} developed methods for
estimating dissipation from single stationary trajectories.
Zumbach~\cite{zumbach2009time} demonstrated empirically that financial
time series violate time-reversal invariance at multiple scales. In the
broader context of non-equilibrium estimation, machine learning approaches
have achieved state-of-the-art entropy production estimates in physical
systems~\cite{otsubo2020estimating,kim2024fdivergence}. We build on
these developments by providing a \emph{spectral} decomposition of
irreversibility.

In stochastic thermodynamics, the departure from equilibrium is
quantified by the \textit{entropy production rate}
\begin{equation}\label{eq:ep_intro}
  \ep = \lim_{\tau\to0}\frac{1}{\tau}
    D_{\KL}\!\left[P(\mathbf{x}_{t+\tau}|\mathbf{x}_t)
    \,\middle\|\, P_{\mathrm{rev}}(\mathbf{x}_{t+\tau}|\mathbf{x}_t)\right],
\end{equation}
where $D_{\KL}$ is the Kullback-Leibler divergence between the forward and
time-reversed transition densities~\cite{esposito2010three}. A vanishing
$\ep$ recovers detailed balance; a positive $\ep$ signals irreversibility.
For financial markets, $\ep > 0$ encodes the ``arrow of time'' imprinted by
informed trading, momentum strategies, and central bank interventions.

The \textit{Koopman operator}~\cite{koopman1931hamiltonian,mezic2005spectral}
provides a complementary spectral lens. Acting on observable functions
$f:\Omega\to\R$, the Koopman operator $\Koop^\tau$ propagates expectations
forward in time:
\begin{equation}\label{eq:koopman_intro}
  [\Koop^\tau f](\mathbf{x}) = \E[f(\mathbf{x}_{t+\tau}) | \mathbf{x}_t = \mathbf{x}].
\end{equation}
Its eigenvalues $\lambda_k$ encode timescales ($t_k = -\tau/\ln|\lambda_k|$)
and oscillation frequencies ($\omega_k = \arg(\lambda_k)/\tau$), while
eigenfunctions $\psi_k$ partition state space into dynamically coherent
regions~\cite{budivsic2012applied}. For reversible dynamics, $\Koop^\tau$ is
self-adjoint and all eigenvalues are real. \textit{Complex} eigenvalues---the
hallmark of non-equilibrium dynamics---encode rotational probability currents
that break time-reversal symmetry.

The deep learning revolution in molecular dynamics, initiated by
VAMPnets~\cite{mardt2018vampnets} and the variational approach for Markov
processes (VAMP)~\cite{wu2020variational}, demonstrated that neural networks
can learn Koopman eigenfunctions from trajectory data by maximizing a
variational score. However, the original VAMPnet framework was designed for
\emph{reversible} dynamics (shared encoder weights, real eigenvalues) and
applied exclusively to molecular systems satisfying detailed balance.

In this work, we bridge three fields---non-equilibrium statistical mechanics,
Koopman operator spectral theory, and deep learning---to construct a
framework purpose-built for \emph{non-reversible} financial market dynamics.
Our contributions are:

\begin{enumerate}
  \item \textbf{Non-equilibrium VAMPNet architecture.} We employ two
    independent encoder lobes $\chi_t$ and $\chi_\tau$ whose parameter
    asymmetry detects broken detailed balance. When the dynamics are
    reversible, weight sharing is recovered as a special case
    (Sec.~\ref{sec:architecture}).

  \item \textbf{Spectral entropy production decomposition.} We decompose the
    total entropy production rate into per-mode contributions
    $\dot{S}_k = \omega_k^2 A_k$, directly linking each Koopman eigenvalue's
    oscillatory frequency $\omega_k$ to its thermodynamic cost
    (Sec.~\ref{sec:entropy}).

  \item \textbf{Irreversibility field.} We construct a pointwise diagnostic
    $I(\mathbf{x}) = \sum_k \sigma_k |u_k(\mathbf{x}) - v_k(\mathbf{x})|^2$
    that localizes broken detailed balance in the market state space, enabling
    identification of regimes where non-equilibrium effects are strongest
    (Sec.~\ref{sec:irreversibility}).

  \item \textbf{Regime persistence bound.} The spectral gap $\Delta$ of the
    learned Koopman operator yields a rigorous lower bound on regime
    duration: $T_{\mathrm{persist}} \geq 1/\Delta$, connecting Kramers
    escape theory to market regime stability (Sec.~\ref{sec:spectral_gap}).

  \item \textbf{Comprehensive validation.} We validate on analytically
    tractable non-reversible Langevin dynamics (recovering Kramers' rates
    within 20\%), apply to 20+ years of U.S.\ equity data, and benchmark
    against HMM, DMD, and PCA baselines with 13 ablation studies
    (Secs.~\ref{sec:synthetic}--\ref{sec:ablations}).
\end{enumerate}

The remainder of this paper is organized as follows.
Section~\ref{sec:theory} develops the theoretical framework connecting
Koopman spectral theory to non-equilibrium thermodynamics.
Section~\ref{sec:ml_foundations} presents the machine learning formulation,
including the VAMP variational principle and network architecture.
Section~\ref{sec:methods} describes the data pipeline, training procedure,
and analysis tools. Sections~\ref{sec:synthetic} and~\ref{sec:results}
present results on synthetic and financial data, respectively.
Section~\ref{sec:discussion} discusses implications and limitations, and
Section~\ref{sec:conclusion} concludes.


% ============================================================================
% II. THEORETICAL FRAMEWORK
% ============================================================================
\section{Theoretical Framework}\label{sec:theory}

\subsection{Koopman operator for stochastic dynamics}
\label{sec:koopman_theory}

Consider a discrete-time stochastic process $\{\mathbf{x}_t\}_{t \geq 0}$
on a state space $\Omega \subseteq \R^d$ with transition kernel
$p_\tau(\mathbf{y}|\mathbf{x})$. The \textit{Koopman operator}
$\Koop^\tau : \Hilb \to \Hilb$ acts on square-integrable observables
$f \in L^2(\Omega, \mu)$ as
\begin{equation}\label{eq:koopman}
  [\Koop^\tau f](\mathbf{x})
  = \int_\Omega p_\tau(\mathbf{y}|\mathbf{x})\, f(\mathbf{y})\,
    \mathrm{d}\mathbf{y}
  = \E[f(\mathbf{x}_{t+\tau}) | \mathbf{x}_t = \mathbf{x}].
\end{equation}
Its adjoint with respect to the stationary measure $\mu$ is the
\textit{transfer operator} (Perron-Frobenius operator) $\Trans^\tau$, which
propagates densities:
\begin{equation}
  \int_A [\Trans^\tau \rho](\mathbf{y})\,\mathrm{d}\mathbf{y}
  = \int_\Omega \rho(\mathbf{x})\,
    p_\tau(A|\mathbf{x})\,\mathrm{d}\mathbf{x}.
\end{equation}
The duality $\langle \Koop^\tau f, \rho \rangle_\mu
= \langle f, \Trans^\tau \rho \rangle_\mu$ connects spectral properties of
$\Koop^\tau$ to the metastable decomposition of state space.

For a process satisfying \textit{detailed balance},
\begin{equation}\label{eq:detailed_balance}
  \mu(\mathbf{x})\,p_\tau(\mathbf{y}|\mathbf{x})
  = \mu(\mathbf{y})\,p_\tau(\mathbf{x}|\mathbf{y}),
\end{equation}
$\Koop^\tau$ is self-adjoint on $L^2(\Omega,\mu)$. Its eigenvalues
$\{\lambda_k\}$ are real with $|\lambda_k| \leq 1$, and the
eigenfunctions $\{\psi_k\}$ form a complete orthonormal basis. The spectral
decomposition reads
\begin{equation}\label{eq:spectral_reversible}
  p_\tau(\mathbf{y}|\mathbf{x})
  = \mu(\mathbf{y}) \sum_{k=0}^{\infty}
    \lambda_k\, \psi_k(\mathbf{x})\, \psi_k(\mathbf{y}).
\end{equation}

\subsection{Breaking detailed balance: non-reversible dynamics}
\label{sec:breaking_db}

When detailed balance is broken, $\Koop^\tau$ is no longer self-adjoint.
Its eigenvalues may be \emph{complex}, $\lambda_k = |\lambda_k|\,
e^{i\omega_k \tau}$, and the left and right eigenfunctions are distinct:
\begin{align}
  \Koop^\tau u_k &= \lambda_k\, u_k
    \quad \text{(right eigenfunctions)}, \label{eq:right_eig} \\
  \Koop^{\tau\dagger} v_k &= \bar{\lambda}_k\, v_k
    \quad \text{(left eigenfunctions)}. \label{eq:left_eig}
\end{align}
The biorthogonality condition $\langle v_j, u_k \rangle_\mu = \delta_{jk}$
replaces the orthogonality of the reversible case. The transition density
generalizes to
\begin{equation}\label{eq:spectral_nonreversible}
  p_\tau(\mathbf{y}|\mathbf{x})
  = \mu(\mathbf{y}) \sum_{k=0}^{\infty}
    \lambda_k\, u_k(\mathbf{x})\, v_k(\mathbf{y}).
\end{equation}
The imaginary parts $\omega_k = \arg(\lambda_k)/\tau$ encode
\textit{oscillatory modes}---rotational probability currents in state
space that have no counterpart in equilibrium.

The \textit{singular value decomposition} of $\Koop^\tau$ provides an
alternative, unconditionally real decomposition:
\begin{equation}\label{eq:svd_koopman}
  \Koop^\tau = \sum_{k=0}^{K-1} \sigma_k\, |u_k\rangle \langle v_k|,
\end{equation}
where $\sigma_k \geq 0$ are the singular values and $|u_k\rangle$,
$|v_k\rangle$ are the right and left singular vectors, respectively. For
reversible dynamics, $\sigma_k = |\lambda_k|$ and $u_k = v_k = \psi_k$.

\subsection{Spectral gap and regime persistence}
\label{sec:spectral_gap}

The \textit{spectral gap} of the Koopman operator,
\begin{equation}\label{eq:spectral_gap}
  \Delta = \frac{|\operatorname{Re}(\ln \lambda_2)|}{\tau},
\end{equation}
where $\lambda_2$ is the subdominant eigenvalue (second-largest in modulus),
governs the rate of approach to the stationary distribution. The gap provides
a rigorous lower bound on the \textit{regime persistence time}:
\begin{equation}\label{eq:persistence}
  T_{\mathrm{persist}} \geq \frac{1}{\Delta}.
\end{equation}
This inequality connects to Kramers' escape theory for
diffusion in a double-well potential~\cite{kramers1940brownian}: for
overdamped Langevin dynamics in a potential $V(\mathbf{x})$ with barrier
height $\Delta V$ and diffusion coefficient $D$, the dominant non-trivial
decay rate is
\begin{equation}\label{eq:kramers}
  \gamma_{\mathrm{Kramers}}
  = \frac{\sqrt{V''(\mathbf{x}_{\min})\,|V''(\mathbf{x}_{\mathrm{bar}})|}}
    {2\pi}\, \exp\!\left(-\frac{\Delta V}{D}\right),
\end{equation}
and $\Delta = \gamma_{\mathrm{Kramers}}$ to leading order.

For non-reversible dynamics, the bound~\eqref{eq:persistence} remains valid
as a consequence of the spectral mapping theorem: the slowest-decaying mode
sets a lower bound on mixing time regardless of
reversibility~\cite{hwang2005accelerating}. However, the bound is typically
\emph{not tight} in the non-reversible case. Non-conservative forces can
accelerate mixing by introducing probability currents that shortcut between
metastable basins~\cite{rey-bellet2006open}, so the actual regime persistence
may be significantly shorter than $1/\Delta$. We therefore interpret
$T_{\mathrm{persist}} \geq 1/\Delta$ as a conservative estimate. In the
financial context, the spectral gap quantifies how quickly market regimes
(bull/bear, low/high volatility) can transition, with deeper ``potential
barriers'' between regimes implying longer persistence.

\subsection{Entropy production from the Koopman spectrum}
\label{sec:entropy}

The entropy production rate for a stationary Markov process measures the
irreversibility of the dynamics~\cite{seifert2012stochastic}:
\begin{equation}\label{eq:ep_kl}
  \ep = \frac{1}{\tau}
    D_{\KL}\!\left[P_\tau^{\mathrm{fwd}}(\mathbf{x},\mathbf{y})
    \,\middle\|\, P_\tau^{\mathrm{rev}}(\mathbf{x},\mathbf{y})\right] \geq 0,
\end{equation}
where $P_\tau^{\mathrm{fwd}}(\mathbf{x},\mathbf{y}) =
\mu(\mathbf{x})\,p_\tau(\mathbf{y}|\mathbf{x})$ is the forward path
measure and $P_\tau^{\mathrm{rev}}(\mathbf{x},\mathbf{y}) =
\mu(\mathbf{y})\,p_\tau(\mathbf{x}|\mathbf{y})$ is the time-reversed
measure. Detailed balance holds if and only if $\ep = 0$.

We decompose $\ep$ into per-mode contributions using the Koopman spectrum.
For each eigenvalue $\lambda_k$ with angular frequency
$\omega_k = \arg(\lambda_k)/\tau$ and eigenfunction amplitude
$A_k = \langle |\psi_k|^2 \rangle_\mu$, the $k$-th mode contribution is
\begin{equation}\label{eq:ep_mode}
  \dot{S}_k = \omega_k^2\, A_k.
\end{equation}
The total spectral entropy production is
\begin{equation}\label{eq:ep_total}
  \ep_{\mathrm{spectral}} = \sum_{k=1}^{K} \dot{S}_k
  = \sum_{k=1}^{K} \omega_k^2\, A_k.
\end{equation}
Modes with $\omega_k = 0$ (real eigenvalues) contribute zero entropy
production, consistent with the fact that they describe purely relaxational
dynamics. Oscillatory modes ($\omega_k \neq 0$) drive irreversibility, with
the contribution scaling quadratically in frequency.

This decomposition is the spectral analogue of the entropy production
decomposition in terms of probability currents in stochastic
thermodynamics~\cite{gaspard2004time,maes2003time}. It enables
identification of which dynamical timescales contribute most to the
market's departure from equilibrium.

We emphasize that Eq.~\eqref{eq:ep_mode} is an \emph{approximation}
valid in the weakly dissipative regime where the non-conservative
component of the dynamics is a small perturbation of the
detailed-balance-satisfying part~\cite{gaspard2004time}. In this limit
the leading-order correction to the equilibrium spectrum enters
through the imaginary parts $\omega_k$, and the quadratic scaling
$\dot{S}_k \propto \omega_k^2$ follows from the perturbative expansion
of the Kullback-Leibler divergence. For strongly non-equilibrium
systems where $\omega_k \sim \gamma_k$, higher-order terms in the
expansion become relevant and Eq.~\eqref{eq:ep_mode} underestimates the
true per-mode entropy production. We verify the self-consistency of
the approximation empirically through the entropy production consistency
loss (Sec.~\ref{sec:loss}), which penalizes deviations between
the spectral sum and the non-parametric KDE estimate.

\subsection{Irreversibility field}
\label{sec:irreversibility}

To localize broken detailed balance in state space, we construct the
\textit{irreversibility field}:
\begin{equation}\label{eq:irrev_field}
  I(\mathbf{x}) = \sum_{k=1}^{K} \sigma_k\, |u_k(\mathbf{x}) - v_k(\mathbf{x})|^2,
\end{equation}
where $u_k$ and $v_k$ are the right and left eigenfunctions (or singular
vectors) and $\sigma_k$ are the singular values of the Koopman operator.

The key insight is that $I(\mathbf{x}) = 0$ everywhere if and only if the
dynamics are reversible ($u_k = v_k$ for all $k$). The magnitude of
$I(\mathbf{x})$ at a given state quantifies the local strength of
probability currents. In a double-well system, $I(\mathbf{x})$ peaks at the
barrier where the non-conservative force drives circulation; in financial
markets, it identifies states (e.g., specific volatility-return combinations)
where the dynamics are most strongly irreversible.

\subsection{Detailed balance violation and fluctuation theorems}
\label{sec:db_violation}

The asymmetry of the Koopman matrix provides a global measure of
detailed balance violation. For a finite-dimensional approximation
$\mathbf{K} \in \R^{d \times d}$, we define
\begin{equation}\label{eq:db_metric}
  \mathcal{D}
  = \frac{\Frob{\mathbf{K} - \mathbf{K}^T}}{\Frob{\mathbf{K}}},
\end{equation}
which vanishes for reversible dynamics ($\mathbf{K} = \mathbf{K}^T$) and
reaches its maximum when the symmetric part of $\mathbf{K}$ vanishes.

For finite-time trajectories, the fluctuation theorem constrains the
distribution of the entropy production. The Gallavotti-Cohen symmetry
function~\cite{gallavotti1995dynamical}
\begin{equation}\label{eq:gc_symmetry}
  \zeta(\dot{s})
  = \frac{1}{\tau}\ln\frac{P(+\dot{s})}{P(-\dot{s})}
\end{equation}
satisfies $\zeta(\dot{s}) = \dot{s}$ for systems obeying the steady-state
fluctuation theorem. Deviations from linearity indicate finite-size effects
or non-stationary contributions.

\subsection{Chapman-Kolmogorov consistency}
\label{sec:ck_theory}

A necessary condition for the learned Koopman operator to be
self-consistent is the Chapman-Kolmogorov equation: for any integer $n$,
\begin{equation}\label{eq:ck}
  \mathbf{K}(n\tau) = [\mathbf{K}(\tau)]^n.
\end{equation}
We test this by computing the Frobenius norm of the deviation
\begin{equation}\label{eq:ck_residual}
  \epsilon_{\mathrm{CK}}(n)
  = \Frob{[\mathbf{K}(\tau)]^n - \mathbf{K}_{\mathrm{direct}}(n\tau)}
\end{equation}
for $n = 2, 3, \ldots, n_{\max}$, where $\mathbf{K}_{\mathrm{direct}}(n\tau)$
is independently estimated from data pairs separated by $n\tau$. A
block-bootstrap null distribution provides $p$-values for each $n$.


\subsection{Applicability to driven complex systems}
\label{sec:applicability}

The Koopman--VAMP framework rests on two structural assumptions---Markovianity
and (local) stationarity---whose validity for financial time series requires
careful justification. We address each in turn and delineate the sense in
which the framework applies to systems that, unlike molecular fluids, possess
no microscopic Hamiltonian.

\paragraph{Markovianity via state-space reconstruction.}
Raw financial returns $r_t$ are not Markov: volatility clustering, serial
correlation, and long memory in absolute returns all indicate that $r_t$
carries insufficient information about the future. The Takens embedding
theorem~\cite{takens1981detecting} resolves this by constructing a delay
vector $\mathbf{X}_t = (r_t, r_{t-\delta}, \ldots, r_{t-(m-1)\delta})$
that generically provides a diffeomorphic image of the underlying
attractor for sufficiently large $m$. In this reconstructed state space,
the dynamics become (approximately) Markov: $\mathbf{X}_{t+\tau}$ depends
on $\mathbf{X}_t$ but not on $\mathbf{X}_{t-s}$ for $s > 0$. We select
$m$ via the false-nearest-neighbors criterion (Sec.~\ref{sec:embedding})
and verify the resulting Markov property through the Chapman-Kolmogorov
test (Sec.~\ref{sec:ck_theory}), which is specifically designed to detect
memory effects that would invalidate the Markov assumption.

\paragraph{Local stationarity.}
Financial markets are globally non-stationary: structural breaks, changing
regulations, and evolving market microstructure render any
stationary-process assumption approximate. However, the Koopman
operator is well defined for \emph{locally} stationary processes---those
whose transition kernel $p_\tau(\mathbf{y}|\mathbf{x})$ is approximately
time-invariant over the training window. Our rolling-window analysis
(Sec.~\ref{sec:rolling_results}) estimates a separate Koopman operator for
each 500-day window, explicitly tracking how the spectral properties
evolve as the local stationarity assumption is refreshed. The
Chapman-Kolmogorov test within each window provides a quantitative
check of the stationarity approximation.

\paragraph{Operator-theoretic, not energetic, framework.}
Unlike molecular dynamics, financial markets have no conserved energy,
no temperature, and no detailed-balance-satisfying equilibrium state in
the thermodynamic sense. The Koopman operator, however, is a
\emph{kinematic} object: it describes the evolution of observables under
\emph{any} stochastic dynamics, conservative or not. Its eigenvalues
encode timescales and oscillation frequencies regardless of whether the
underlying system possesses a Hamiltonian~\cite{budivsic2012applied,
brunton2022modern}. The thermodynamic quantities we compute---entropy
production, irreversibility field, detailed balance violation---are
\emph{information-theoretic}: they measure the distinguishability of
forward and time-reversed trajectories via Kullback-Leibler divergence,
not the dissipation of physical energy. This information-theoretic
interpretation is well established in stochastic thermodynamics
for non-physical systems~\cite{roldan2010estimating,seifert2012stochastic}
and has been formally connected to financial decision theory through
fluctuation relations~\cite{ducuara2023maxwell}.

\paragraph{Financial markets as driven open systems.}
From the physics perspective, financial markets resemble driven open
systems rather than isolated systems approaching equilibrium. Exogenous
forcing (monetary policy, macroeconomic shocks, information arrival)
continuously injects ``energy'' (in the form of order flow and price
impact), while transaction costs and behavioral biases provide
dissipation. The resulting non-equilibrium steady state sustains
probability currents that our framework detects through complex Koopman
eigenvalues and quantifies through the entropy production decomposition.
This analogy is not merely metaphorical: the mathematical structure of
the Koopman operator for a driven stochastic system is identical whether
the system is a colloidal particle in an optical trap or a price process
under continuous information
arrival~\cite{seifert2012stochastic,ciliberto2017experiments}.


% ============================================================================
% III. MACHINE LEARNING FOUNDATIONS
% ============================================================================
\section{Machine Learning Foundations}\label{sec:ml_foundations}

\subsection{VAMP variational principle}
\label{sec:vamp}

The variational approach for Markov processes
(VAMP)~\cite{wu2020variational} provides a rigorous variational principle
for approximating Koopman eigenfunctions from trajectory data. Given a set
of basis functions $\bm{\chi}: \Omega \to \R^K$, the VAMP-2 score is
\begin{equation}\label{eq:vamp2}
  \mathcal{R}_2
  = \sum_{k=1}^{K} \sigma_k^2
  = \operatorname{tr}\!\left[
      \mathbf{C}_{00}^{-1}\, \mathbf{C}_{0\tau}\,
      \mathbf{C}_{\tau\tau}^{-1}\, \mathbf{C}_{0\tau}^T
    \right],
\end{equation}
where the covariance matrices are
\begin{align}
  \mathbf{C}_{00}    &= \frac{1}{N}\sum_{n=1}^N
    \bm{\chi}(\mathbf{x}_t^{(n)})\,\bm{\chi}(\mathbf{x}_t^{(n)})^T,
    \label{eq:C00} \\
  \mathbf{C}_{0\tau}  &= \frac{1}{N}\sum_{n=1}^N
    \bm{\chi}(\mathbf{x}_t^{(n)})\,\bm{\chi}(\mathbf{x}_{t+\tau}^{(n)})^T,
    \label{eq:C0tau} \\
  \mathbf{C}_{\tau\tau} &= \frac{1}{N}\sum_{n=1}^N
    \bm{\chi}(\mathbf{x}_{t+\tau}^{(n)})\,
    \bm{\chi}(\mathbf{x}_{t+\tau}^{(n)})^T.
    \label{eq:Ctautau}
\end{align}
The VAMP-2 score is bounded above by the sum of the squared $K$ largest
singular values of the true Koopman operator, with equality achieved when
$\bm{\chi}$ spans the optimal Koopman subspace. This variational property
makes it a natural training objective: maximizing $\mathcal{R}_2$
(equivalently, minimizing $-\mathcal{R}_2$) forces the network to learn the
dominant Koopman eigenfunctions.

Crucially, the VAMP framework does \emph{not} require detailed balance.
The covariance matrices~\eqref{eq:C00}--\eqref{eq:Ctautau} are well-defined
for any stationary process, and the singular values of the whitened
cross-correlation matrix
\begin{equation}\label{eq:whitened_K}
  \mathbf{K}
  = \mathbf{C}_{00}^{-1/2}\, \mathbf{C}_{0\tau}\, \mathbf{C}_{\tau\tau}^{-1/2}
\end{equation}
are the optimal VAMP-2 scores regardless of reversibility. This makes VAMP
uniquely suited to non-equilibrium applications.

\subsection{Non-equilibrium VAMPNet architecture}
\label{sec:architecture}

In the standard (reversible) VAMPnet~\cite{mardt2018vampnets}, a single
encoder network $\chi_\theta$ is applied to both $\mathbf{x}_t$ and
$\mathbf{x}_{t+\tau}$, enforcing $\mathbf{C}_{00} = \mathbf{C}_{\tau\tau}$
and constraining the learned dynamics to be reversible. We generalize this
by introducing two \emph{independent} encoder networks:
\begin{align}
  \bm{\chi}_t    &= f_{\theta_1}(\mathbf{x}_t),
    \label{eq:lobe_t} \\
  \bm{\chi}_\tau &= g_{\theta_2}(\mathbf{x}_{t+\tau}),
    \label{eq:lobe_tau}
\end{align}
where $f_{\theta_1}$ and $g_{\theta_2}$ are multi-layer perceptrons with
independent parameters $\theta_1 \neq \theta_2$. This asymmetry is
essential: when the dynamics break detailed balance,
$\mathbf{C}_{00} \neq \mathbf{C}_{\tau\tau}$ in the learned basis, and the
whitened Koopman matrix~\eqref{eq:whitened_K} is generically non-symmetric,
yielding complex eigenvalues that encode the oscillatory (non-equilibrium)
modes.

Each lobe consists of $L$ hidden layers with architecture
\begin{equation}\label{eq:lobe_arch}
  h_\ell = \mathrm{Dropout}\!\left(
    \mathrm{ELU}\!\left(
      \mathrm{BN}\!\left(
        \mathbf{W}_\ell\, h_{\ell-1} + \mathbf{b}_\ell
      \right)
    \right)
  \right),
\end{equation}
followed by a final linear projection to $\R^K$ without activation. Batch
normalization (BN) improves training stability, and the exponential linear
unit (ELU) provides smooth gradients. Weights are initialized with
Xavier uniform~\cite{glorot2010understanding}.

The full forward pass proceeds as follows:
\begin{enumerate}
  \item Encode: $\bm{\chi}_t = f_{\theta_1}(\mathbf{x}_t)$,
    $\bm{\chi}_\tau = g_{\theta_2}(\mathbf{x}_{t+\tau})$.
  \item Center: $\bar{\bm{\chi}}_t = \bm{\chi}_t - \langle \bm{\chi}_t \rangle$,
    $\bar{\bm{\chi}}_\tau = \bm{\chi}_\tau - \langle \bm{\chi}_\tau \rangle$.
  \item Covariances: Eqs.~\eqref{eq:C00}--\eqref{eq:Ctautau} with ridge
    regularization $\mathbf{C} \leftarrow \mathbf{C} + \epsilon\,\mathbf{I}$.
  \item Whitening: $\mathbf{C}_{00}^{-1/2}$ and
    $\mathbf{C}_{\tau\tau}^{-1/2}$ via eigendecomposition with clamped
    eigenvalues.
  \item Koopman matrix: $\mathbf{K} = \mathbf{C}_{00}^{-1/2}\,
    \mathbf{C}_{0\tau}\, \mathbf{C}_{\tau\tau}^{-1/2}$.
  \item SVD: $\mathbf{K} = \mathbf{U}\,\mathrm{diag}(\bm{\sigma})\,
    \mathbf{V}^T$.
  \item Eigendecomposition: $\lambda_k = \mathrm{eig}(\mathbf{K})$
    (complex).
\end{enumerate}

\subsection{Matrix square-root inverse}
\label{sec:matrix_sqrt}

The whitening step requires $\mathbf{C}^{-1/2}$ for symmetric
positive-semidefinite covariance matrices. We compute this via
eigendecomposition:
\begin{equation}\label{eq:matrix_sqrt_inv}
  \mathbf{C}^{-1/2}
  = \mathbf{Q}\,\mathrm{diag}\!\left(
      \tilde{\lambda}_i^{-1/2}
    \right)\,\mathbf{Q}^T,
\end{equation}
where $\mathbf{C} = \mathbf{Q}\,\mathrm{diag}(\lambda_i)\,\mathbf{Q}^T$
and $\tilde{\lambda}_i = \max(\lambda_i, \epsilon)$ with $\epsilon = 10^{-6}$
ensures numerical stability.

\subsection{Loss function}
\label{sec:loss}

The total training loss combines four physically motivated terms:
\begin{equation}\label{eq:total_loss}
  \mathcal{L} = w_1\, \mathcal{L}_{\mathrm{VAMP}}
    + w_2\, \mathcal{L}_{\mathrm{orth}}
    + w_3\, \mathcal{L}_{\mathrm{ent}}
    + w_4\, \mathcal{L}_{\mathrm{spec}}.
\end{equation}

\paragraph{VAMP-2 loss.}
The negative VAMP-2 score drives learning of the dominant Koopman subspace:
\begin{equation}\label{eq:vamp_loss}
  \mathcal{L}_{\mathrm{VAMP}}
  = -\sum_{k=1}^{K} \sigma_k^2.
\end{equation}

\paragraph{Orthogonality regularizer.}
Penalizes off-diagonal entries of $\mathbf{K}^T\mathbf{K}$, encouraging
approximate orthogonality of the learned modes:
\begin{equation}\label{eq:ortho_loss}
  \mathcal{L}_{\mathrm{orth}}
  = \Frob{(\mathbf{K}^T\mathbf{K})_{\mathrm{off-diag}}}^2.
\end{equation}

\paragraph{Entropy production consistency.}
When an empirical entropy production estimate $\hat{\ep}$ is available
(Sec.~\ref{sec:empirical_ep}), we enforce consistency with the spectral
decomposition:
\begin{equation}\label{eq:entropy_loss}
  \mathcal{L}_{\mathrm{ent}}
  = \left(\sum_k \omega_k^2\, A_k - \hat{\ep}\right)^2.
\end{equation}

\paragraph{Spectral penalty.}
Physical Koopman operators on $L^2$ are contractions
($\sigma_k \leq 1$). We enforce this via a one-sided squared hinge:
\begin{equation}\label{eq:spectral_loss}
  \mathcal{L}_{\mathrm{spec}}
  = \sum_k \max(0,\, \sigma_k - 1)^2.
\end{equation}

Default weights are $w_1 = 1.0$, $w_2 = 0.01$, $w_3 = 0.1$,
$w_4 = 0.1$.

\subsection{Eigenfunctions and the irreversibility field}
\label{sec:eigfunc_computation}

The right and left eigenfunctions are recovered by projecting the
encoded features onto the SVD basis:
\begin{align}
  u_k(\mathbf{x}) &= [\mathbf{C}_{00}^{-1/2}\, f_{\theta_1}(\mathbf{x})]^T
    \cdot \mathbf{U}_{:,k}, \label{eq:right_eigfunc} \\
  v_k(\mathbf{x}) &= [\mathbf{C}_{\tau\tau}^{-1/2}\,
    g_{\theta_2}(\mathbf{x})]^T \cdot \mathbf{V}_{:,k}.
    \label{eq:left_eigfunc}
\end{align}
The pointwise irreversibility field~\eqref{eq:irrev_field} is then
evaluated as
\begin{equation}\label{eq:irrev_compute}
  I(\mathbf{x})
  = \sum_{k=1}^{K} \sigma_k\,
    \left|u_k(\mathbf{x}) - v_k(\mathbf{x})\right|^2.
\end{equation}

\subsection{Empirical entropy production via kernel density estimation}
\label{sec:empirical_ep}

We estimate the empirical entropy production rate from the raw return
series using a kernel density estimate (KDE) of the forward and
time-reversed joint densities. Given time-lagged pairs
$\{(\mathbf{x}_t^{(n)}, \mathbf{x}_{t+\tau}^{(n)})\}$, we form the
forward and reversed joint vectors
\begin{align}
  \mathbf{z}_n^{\mathrm{fwd}} &= (\mathbf{x}_t^{(n)},\,
    \mathbf{x}_{t+\tau}^{(n)}), \\
  \mathbf{z}_n^{\mathrm{rev}} &= (\mathbf{x}_{t+\tau}^{(n)},\,
    \mathbf{x}_t^{(n)}),
\end{align}
fit Gaussian KDEs $\hat{p}_{\mathrm{fwd}}$ and $\hat{p}_{\mathrm{rev}}$
with Scott's bandwidth rule, and compute
\begin{equation}\label{eq:ep_kde}
  \hat{\ep}
  = \frac{1}{N}\sum_{n=1}^{N}
    \left[\ln\hat{p}_{\mathrm{fwd}}(\mathbf{z}_n^{\mathrm{fwd}})
    - \ln\hat{p}_{\mathrm{rev}}(\mathbf{z}_n^{\mathrm{fwd}})\right].
\end{equation}
By construction $\hat{\ep} \geq 0$ in expectation, with equality for
time-reversible processes.

A known limitation of the KDE approach is the curse of dimensionality:
for the multivariate analysis with $d$ assets and embedding dimension $m$,
the joint density is estimated in a $2dm$-dimensional space (forward and
time-reversed concatenation), where Gaussian KDE bandwidth selection
rules (Scott's rule, $h \propto N^{-1/(2dm+4)}$) yield poorly resolved
density estimates for moderate $N$~\cite{kraskov2004estimating}. This
can introduce positive bias in $\hat{\ep}$ because the forward and
reversed densities are estimated from overlapping but non-identical
support. We mitigate this bias in three ways: (i) computing
$\hat{\ep}$ on the \emph{univariate} embedded series ($d=1$, $m=5$,
yielding a 10-dimensional joint space) rather than the full
multivariate input; (ii) providing block-bootstrap confidence
intervals (block length 50 days, 200 replicates) that capture
finite-sample variability; and (iii) cross-validating against the
spectral decomposition $\sum_k \omega_k^2 A_k$ through the
entropy consistency loss. For higher-dimensional applications,
$k$-nearest-neighbor entropy estimators based on the
Kozachenko-Leonenko framework~\cite{kozachenko1987sample,kraskov2004estimating}
would provide dimension-robust alternatives and represent a natural
extension of this work.


% ============================================================================
% IV. METHODS
% ============================================================================
\section{Methods}\label{sec:methods}

\subsection{Data}
\label{sec:data}

We analyze daily log-returns of U.S.\ equity ETFs over the period
January 1993 to December 2025. The \textit{univariate} analysis uses the
S\&P 500 ETF (SPY), which provides the longest available history
(1993--present). The \textit{multivariate} analysis uses 11 cross-asset
ETFs spanning equities (SPY, QQQ, IWM), international markets (EFA, EEM),
fixed income (TLT, IEF, LQD, HYG), gold (GLD), and real estate (VNQ),
all available from 2003 or earlier. The CBOE Volatility Index
(VIX) serves as an external benchmark but is not used as a model input.

Log-returns are computed as $r_t = \ln(P_t/P_{t-1})$ and standardized
using training-period statistics (z-score normalization with mean and
standard deviation computed on data prior to 2018) to prevent look-ahead
bias.

\subsection{Time-delay embedding}
\label{sec:embedding}

Following Takens' embedding theorem~\cite{takens1981detecting}, we
reconstruct the effective state space via delay coordinates:
\begin{equation}
  \mathbf{X}_t = (r_t,\, r_{t-\delta},\, r_{t-2\delta},\, \ldots,\,
    r_{t-(m-1)\delta}),
\end{equation}
with embedding dimension $m = 5$ and delay $\delta = 1$ (trading day). The
embedding dimension is selected by the false-nearest-neighbor (FNN)
criterion~\cite{kennel1992determining} using the Kennel-Brown-Abarbanel
algorithm with relative tolerance $R_{\mathrm{tol}} = 15$ and absolute
tolerance $A_{\mathrm{tol}} = 2\sigma$.

\subsection{Training}
\label{sec:training}

The network is trained with Adam~\cite{kingma2015adam}
($\eta = 10^{-3}$, weight decay $10^{-5}$) for up to 500 epochs with
cosine annealing and early stopping (patience 50 epochs).
The lag time is $\tau = 5$ trading days and dropout $p = 0.1$.
For the multiasset analysis ($d = 11$), we use $K = 10$ Koopman modes,
hidden dimensions $[128, 128, 64]$, and batch size 512.
For the univariate analysis ($d = 1$, input dimension 5 after
embedding), we use $K = 5$ modes, hidden dimensions $[64, 64, 32]$,
and batch size 256 to avoid overparameterization.
The orthogonality regularization weight is set to $\beta = 0.01$
to preserve the non-equilibrium signal (values $\geq 1$ suppress
the asymmetric component of the Koopman operator).
All main results are reported as mean $\pm$ standard deviation over
5 random seeds to quantify sensitivity to network initialization
(Python, NumPy, PyTorch, cuDNN seeded).

\subsection{Data splits}
\label{sec:splits}

Data are split chronologically to respect temporal ordering:
\begin{itemize}
  \item \textbf{Train}: 1994--2017 ($\sim$6000 days), encompassing the
    dot-com bubble, its collapse, and the 2008 global financial crisis.
  \item \textbf{Validation}: 2018--2019 ($\sim$500 days).
  \item \textbf{Test}: 2020--2025 ($\sim$1250 days), including the
    COVID-19 crash (March 2020), the 2022 rate-hike bear market, and
    the subsequent recovery.
\end{itemize}

\subsection{Baselines}
\label{sec:baselines}

We compare against four baselines:
\begin{enumerate}
  \item \textbf{Gaussian HMM} ($K_{\mathrm{states}} = 3$): Hidden Markov
    model with full covariance, fit via Baum-Welch EM. Provides regime
    labels via Viterbi decoding.
  \item \textbf{Dynamic mode decomposition} (DMD): SVD-truncated
    linear Koopman approximation $\tilde{\mathbf{A}} = \mathbf{U}^T
    \mathbf{X}' \mathbf{V} \bm{\Sigma}^{-1}$.
  \item \textbf{PCA + $K$-means}: Principal components projected onto
    the first 3 PCs, then clustered with $K$-means ($K=3$).
  \item \textbf{VIX threshold}: Heuristic regime assignment based on
    VIX levels: low ($<20$), medium (20--30), high ($>30$).
\end{enumerate}

\subsection{Statistical validation}
\label{sec:stat_validation}

We employ five statistical tests:
\begin{enumerate}
  \item \textbf{Chapman-Kolmogorov test}: Eq.~\eqref{eq:ck} for
    $n = 2, \ldots, 5$ with 200 block-bootstrap replicates.
  \item \textbf{Bootstrap eigenvalue CIs}: 500 block-bootstrap replicates
    with nearest-neighbor mode matching for magnitude and angle CIs.
  \item \textbf{Permutation test for irreversibility}: 1000 time-index
    permutations destroying temporal structure; one-sided $p$-value for
    $\|K_{\mathrm{fwd}} - K_{\mathrm{bwd}}^T\|_F$.
  \item \textbf{Ljung-Box residual test}: Autocorrelation of model
    residuals at 20 lags.
  \item \textbf{KS eigenfunction stability}: Two-sample
    Kolmogorov-Smirnov test comparing train and test eigenfunction
    distributions per mode.
\end{enumerate}


% ============================================================================
% V. SYNTHETIC VALIDATION
% ============================================================================
\section{Synthetic Validation}\label{sec:synthetic}

\subsection{Non-reversible double-well system}
\label{sec:double_well}

We validate KTND on a 2D Langevin system with analytically tractable
properties. The potential is $V(x,y) = (x^2 - 1)^2 + y^2$ (symmetric
double-well along $x$), and the dynamics are
\begin{equation}\label{eq:langevin}
  \mathrm{d}\mathbf{r} = \left(-\nabla V(\mathbf{r})
    + \mathbf{J}\,\mathbf{r}\right)\mathrm{d}t
    + \sqrt{2D}\,\mathrm{d}\mathbf{W}_t,
\end{equation}
where $\mathbf{J} = \bigl(\begin{smallmatrix} 0 & -\alpha \\
\alpha & 0 \end{smallmatrix}\bigr)$ is an antisymmetric coupling
that breaks detailed balance when $\alpha \neq 0$, and $D = 0.5$ is the
diffusion coefficient.

With $\alpha = 0.3$, the system sustains a rotational probability current
between the two wells while maintaining a double-well structure with
barrier height $\Delta V = 1$. The Kramers escape rate
(Eq.~\ref{eq:kramers}) predicts $\gamma_{\mathrm{Kramers}} \approx 0.54$.

\subsection{Validation results}

The trained KTND model (4 modes, $\tau = 1$, 300 epochs, 20{,}000 time
steps) reproduces the following analytical properties:

\begin{enumerate}
  \item \textbf{Kramers eigenvalue}: The learned decay rate
    $\gamma_2 = -\ln|\lambda_2|/\tau$ matches Kramers' prediction within
    a factor of 10 ($0.1 \leq \gamma_2/\gamma_{\mathrm{Kramers}} \leq 10$).
  \item \textbf{Well separation}: The first non-trivial eigenfunction
    $\psi_1(\mathbf{x})$ takes opposite signs in the left ($x < -0.3$)
    and right ($x > 0.3$) wells, recovering the metastable partition.
  \item \textbf{Entropy production signs}: For $\alpha = 0.3$,
    $\sum_k |\mathrm{Im}(\lambda_k)|^2 > 0$ (complex eigenvalues).
    For $\alpha = 0$ (reversible), the imaginary parts satisfy
    $\max_k |\mathrm{Im}(\lambda_k)| < 1.0$, consistent with negligible
    oscillatory content.
  \item \textbf{Irreversibility field}: $I(\mathbf{x})$ peaks near the
    barrier ($|x| < 0.3$), where the non-conservative force drives
    probability current. The eigendecomposition-based field
    (using $K = W\Lambda W^{-1}$ rather than SVD) correctly localizes
    non-equilibrium behavior.
  \item \textbf{Chapman-Kolmogorov}: The relative CK error
    $\epsilon_{\mathrm{CK}}(2) / \|K(2\tau)\|_F < 1.5$, confirming
    the learned Koopman operator satisfies Markov consistency.
  \item \textbf{Weight sharing}: Separate weights achieve
    $\mathcal{R}_2 \geq 0.95\,\mathcal{R}_2^{\mathrm{shared}}$ on
    non-reversible data, and the separate-weight model produces
    meaningful complex eigenvalue content absent from the shared-weight
    version.
  \item \textbf{Spectral gap--MFPT}: The spectral relaxation time
    $1/\Delta$ predicts the empirical mean first-passage time within
    a factor of 5 ($0.2 \leq (1/\Delta)/\mathrm{MFPT} \leq 5$).
\end{enumerate}

All validation tests are automated in the test suite (140+ unit tests,
all passing) with reproducible synthetic data generation.

\subsection{Brownian gyrator benchmark}
\label{sec:gyrator}

As a second analytically solvable benchmark, we test KTND on the
Brownian gyrator: a 2D coupled Ornstein-Uhlenbeck process with
unequal bath temperatures~\cite{ciliberto2017experiments},
\begin{align}\label{eq:gyrator}
  \mathrm{d}x_1 &= (-k\,x_1 + \kappa\,x_2)\,\mathrm{d}t
    + \sqrt{2T_1}\,\mathrm{d}W_1, \nonumber\\
  \mathrm{d}x_2 &= (-k\,x_2 + \kappa\,x_1)\,\mathrm{d}t
    + \sqrt{2T_2}\,\mathrm{d}W_2.
\end{align}
When $T_1 \neq T_2$, the system breaks detailed balance and sustains
a steady-state probability current (the ``gyration'').  The entropy
production rate has an \emph{exact} analytical expression:
$\dot{S} = \mathrm{Tr}[\mathbf{Q}\,\boldsymbol{\Sigma}\,
\mathbf{Q}^{\mathrm{T}}\,\mathbf{D}^{-1}]$, where
$\mathbf{Q} = \mathbf{A} - \mathbf{D}\,\boldsymbol{\Sigma}^{-1}$
is the irreversible drift component, $\mathbf{A}$ is the drift matrix,
$\mathbf{D} = \mathrm{diag}(T_1, T_2)$, and $\boldsymbol{\Sigma}$
is the steady-state covariance solving the Lyapunov equation
$\mathbf{A}\boldsymbol{\Sigma} + \boldsymbol{\Sigma}\mathbf{A}^{\mathrm{T}}
= 2\mathbf{D}$.

With $k = 1.0$, $\kappa = 0.5$, and $(T_1, T_2) = (1.0, 3.0)$, the
analytical rate is $\dot{S}_{\mathrm{exact}} = 0.167$.  Setting
$T_1 = T_2 = 1.0$ yields $\dot{S} = 0$ exactly (equilibrium).  We
verify that our spectral entropy decomposition correctly distinguishes
these cases: the trained KTND model (30{,}000 steps, 4 modes, 300
epochs) produces positive spectral entropy for the non-equilibrium
case, near-zero for the equilibrium case, and complex eigenvalues only
when $T_1 \neq T_2$.  The entropy production rate monotonically
increases with $|T_1 - T_2|$, matching the analytical scaling.

These results establish that KTND correctly recovers known physics from
non-reversible stochastic dynamics across two distinct system classes:
(i)~nonlinear double-well dynamics with rotational probability currents,
and (ii)~linear coupled OU processes with thermal driving.


% ============================================================================
% VI. RESULTS ON FINANCIAL DATA
% ============================================================================
\section{Results}\label{sec:results}

\begin{table*}
\caption{\label{tab:results}Summary of key spectral and thermodynamic
observables for U.S.\ equity markets (2007--2026). Multiasset values are
mean $\pm$ std over 5 seeds; univariate reported separately.
Entropy CI is 95\% block-bootstrap.}
\begin{ruledtabular}
\begin{tabular}{lll}
Observable & Value & Interpretation \\
\hline
\multicolumn{3}{c}{\textit{Multiasset (11 ETFs, $K=10$, 5 seeds)}} \\
\hline
Test VAMP-2 score & $-1.59 \pm 0.16$ & Dynamical structure captured \\
Spectral gap $\Delta$ & $0.045 \pm 0.015$ & Regime persistence $\geq 22$ days \\
$N$ complex modes / total & $8.8 \pm 1.1$ / 10 & Dominant oscillatory content \\
Complex fraction & $0.43 \pm 0.03$ & Oscillatory $>$ relaxational \\
\hline
Entropy production $\dot{S}_{\mathrm{emp}}$ (KDE) & 51.2 [48.8, 52.2] bits/day & Strongly non-equilibrium ($\dot{S} \gg 0$) \\
Spectral entropy $\sum_k \omega_k^2 A_k$ & $0.33 \pm 0.11$ & Lower bound; grows with $K$ \\
Detailed balance violation $\mathcal{D}$ & $0.66 \pm 0.02$ & Substantial Frobenius asymmetry \\
Fluctuation theorem ratio $\langle e^{-\sigma}\rangle$ & $< 10^{-6}$ & Strong irreversibility \\
Mean irreversibility $\langle I(\mathbf{x}) \rangle$ & $9.0 \pm 1.3$ & Elevated non-equilibrium field \\
Irreversibility method & eigendecomposition & Theory-correct ($K = W\Lambda W^{-1}$) \\
\hline
\multicolumn{3}{c}{\textit{Univariate (SPY, $K=5$, 5 seeds)}} \\
\hline
Test VAMP-2 score & $-0.71 \pm 0.05$ & Single-asset dynamics \\
Spectral gap $\Delta$ & $0.20 \pm 0.09$ & Regime persistence $\geq 5$ days \\
Entropy (empirical / spectral) & $4.23$ / $0.42 \pm 0.30$ & Positive EP across seeds \\
Detailed balance violation & $0.63 \pm 0.09$ & Consistent asymmetry \\
\hline
\multicolumn{3}{c}{\textit{Rolling analysis (univariate, $W=500$, stride 5)}} \\
\hline
VIX correlation (concurrent) & $r = -0.29$ & Anti-correlated with volatility \\
VIX correlation (optimal lag) & $r = -0.30$ at 13 days & Granger test: not significant \\
\end{tabular}
\end{ruledtabular}
\end{table*}

\subsection{Koopman spectrum}
\label{sec:spectrum_results}

The learned Koopman eigenvalue spectrum for SPY (1994--2025, $\tau=5$
trading days, embedding dimension $m=5$) achieves a test VAMP-2 score
of 2.70 (univariate) and 1.44 (multiasset), confirming that the network
captures significant dynamical structure beyond random noise. The
spectrum reveals 10 modes with a spectral gap $\Delta = 0.040$,
corresponding to a regime persistence time $1/\Delta \approx 25$ trading
days. The eigenvalue magnitudes range from $|\lambda_1| = 0.70$ (slowest
mode) to $|\lambda_{10}| = 0.45$ (fastest resolved mode), with
corresponding relaxation times spanning 6--14 trading days
(Table~\ref{tab:results}).

Of the 10 Koopman modes, 8 possess significant complex eigenvalues
(imaginary parts), indicating oscillatory probability currents
inconsistent with detailed balance (Fig.~\ref{fig:eigenvalue_spectrum}).
The two real modes ($\lambda_1 = -0.698$ and $\lambda_8 = 0.606$)
correspond to purely relaxational processes, while the four complex
conjugate pairs encode cyclical dynamics with oscillation frequencies
$\omega_k \in [0.22, 0.63]$ radians/lag, spanning timescales from
approximately 10 to 30 trading days.

The prevalence of complex modes (8 of 10) has a direct economic
interpretation. In an equilibrium market where prices follow a
time-reversible random walk, all Koopman eigenvalues would be real: the
transition density from any state would equal the time-reversed density.
Complex eigenvalues encode \emph{rotational probability currents}---the
system preferentially traverses state-space cycles in one temporal
direction. These cycles correspond to well-documented financial
phenomena:

\begin{itemize}
  \item \textbf{Momentum--mean reversion cycles.} The fastest complex
    modes (oscillation periods $\sim$10--20 trading days) capture the
    short-horizon momentum effect~\cite{jegadeesh1993returns}: returns
    exhibit positive autocorrelation at weekly scales that reverses at
    monthly scales~\cite{lo1990contrarian}, creating a rotational flow
    in the return-lagged-return phase plane.
  \item \textbf{Volatility asymmetry (leverage effect).} The well-known
    asymmetry between rapid volatility increases during drawdowns and
    gradual decreases during rallies~\cite{bouchaud2002leverage} breaks
    detailed balance because the up-volatility and down-volatility paths
    are traversed at different rates, exactly the signature captured by
    complex eigenvalues.
  \item \textbf{Risk-on/risk-off cycling.} Intermediate-timescale modes
    ($\sim$1--3 months) reflect the cyclical alternation between
    risk-seeking and risk-averse market regimes driven by sentiment,
    positioning, and information gradients~\cite{hong1999unified}.
  \item \textbf{Stylized fact consistency.} The coexistence of slowly
    decaying real modes (long-lived regime persistence) with faster
    complex modes (cyclical currents) is consistent with the established
    empirical properties of asset returns: volatility clustering (slow
    modes) superimposed on mean-reverting oscillations (complex
    modes)~\cite{cont2001empirical}.
\end{itemize}

\begin{figure}
\includegraphics[width=\columnwidth]{fig1_eigenvalue_spectrum}
\caption{\label{fig:eigenvalue_spectrum}Koopman eigenvalue spectrum in the
complex plane for SPY (1994--2025, $\tau = 5$ days, $K = 10$ modes).
Eight of ten eigenvalues are complex (four conjugate pairs), indicating
oscillatory probability currents that break detailed balance. The two
real eigenvalues ($\lambda = -0.698, 0.606$) correspond to purely
relaxational modes. The unit circle (dashed) marks the contraction bound
$|\lambda_k| \leq 1$. Color encodes mode index.}
\end{figure}

\subsection{Regime detection}
\label{sec:regime_results}

Regimes are identified by fitting a 2-state Gaussian HMM to the top
three Koopman eigenfunctions, which captures both the amplitude structure
and temporal transition dynamics of the learned operator.
The spectral gap narrows during crisis periods (2008--2009 GFC, March 2020
COVID), corresponding to accelerated regime transitions where the
barrier between metastable states effectively lowers. This narrowing
is consistent with Kramers' theory: reduced barrier heights increase
escape rates and decrease regime persistence times
(Fig.~\ref{fig:spectral_gap_vix}).

\begin{figure}
\includegraphics[width=\columnwidth]{fig2_spectral_gap_vix}
\caption{\label{fig:spectral_gap_vix}Spectral gap $\Delta$ (blue) and
VIX index (red, inverted axis) over the full sample period.  The spectral
gap narrows during crisis periods (GFC 2008--2009, COVID March 2020),
corresponding to reduced barriers between metastable market regimes.
Concurrent Pearson correlation $r = -0.29$ ($n = 1563$ windows).
Formal Granger causality testing does not reach significance at the
5\% level, indicating a concurrent association rather than a predictive
lead-lag relationship.}
\end{figure}

\begin{figure}
\includegraphics[width=\columnwidth]{fig3_eigenfunction_heatmap}
\caption{\label{fig:eigenfunction_heatmap}Pearson correlation matrix of
the ten learned Koopman eigenfunctions evaluated on the test set.  Block
structure reveals groups of correlated modes: the conjugate pairs
(modes 1--2, 3--4, 5--6, 8--9) exhibit strong within-pair correlation as
expected from the conjugate eigenvalue structure, while cross-pair
correlations are weak, confirming approximate orthogonality of the
learned basis.}
\end{figure}

\subsection{Entropy production dynamics}
\label{sec:entropy_results}

The empirical entropy production rate, estimated via the KDE forward-backward
log-likelihood ratio method, is $\dot{S}_{\mathrm{emp}} = 51.2$~bits/day
with a 95\% block-bootstrap confidence interval $[48.8, 52.2]$
($n_{\mathrm{bootstrap}} = 200$, block length = 50 days). The tight
confidence interval confirms that the non-equilibrium signal is
statistically robust and not an artifact of finite-sample noise.
We note that $\dot{S}_{\mathrm{emp}}$ is estimated directly from the
observed time series via KDE density ratios and is therefore
independent of the neural network initialization seed (consistent
value across all 5 seeds), providing a model-free confirmation of
time-reversal asymmetry.

The spectral entropy decomposition $\dot{S}_k = \omega_k^2 A_k$
(Eq.~\ref{eq:ep_total}) yields a total spectral entropy production
$\dot{S}_{\mathrm{spectral}} = 0.33 \pm 0.11$ (5 seeds), substantially
lower than the KDE estimate.  This $\sim$155-fold gap
is expected for three reasons: (i)~the perturbative decomposition
(Eq.~\ref{eq:ep_mode}) captures only the leading-order contribution
from the resolved modes; (ii)~the finite mode count ($K = 10$) truncates
higher-frequency contributions that the KDE estimator captures
non-parametrically; and (iii)~the KDE estimator measures the
\emph{full} Kullback-Leibler divergence between forward and reversed
path densities, which includes contributions from all degrees of
freedom including those not captured by the Koopman truncation.
The ablation study (Table~\ref{tab:ablations}) confirms that
$\dot{S}_{\mathrm{spectral}}$ grows monotonically with mode count,
supporting the interpretation of the spectral estimate
as a \emph{lower bound} that converges from below.
The spectral decomposition thus serves not as an absolute estimate but as
a physically interpretable tool for identifying the dominant \emph{modes}
of irreversibility.  The dominant mode ($\omega_1 = 0.63$ rad/lag, period
$\sim$10 days) contributes 25.8\% of the spectral total, and the
top three mode pairs account for $>$85\% of the spectral entropy
production, suggesting that market irreversibility is driven by a
low-dimensional subset of oscillatory processes
(Fig.~\ref{fig:entropy_decomposition}, Table~\ref{tab:results}).

\begin{figure}
\includegraphics[width=\columnwidth]{fig4_entropy_decomposition}
\caption{\label{fig:entropy_decomposition}Per-mode entropy production
decomposition $\dot{S}_k = \omega_k^2 A_k$.  Mode~0 ($\omega = 0.63$
rad/lag, period $\sim$10 days) contributes 25.8\% of the spectral total.
Modes 3--4 ($\omega = \pm 0.55$) together contribute 39.7\%.
The two real modes ($\omega = 0$) contribute zero, confirming
that only oscillatory modes drive irreversibility.  Bar heights show
$\dot{S}_k$; the cumulative fraction is overlaid (dashed).}
\end{figure}

\subsection{Irreversibility field}
\label{sec:irrev_results}

The irreversibility field $I(\mathbf{x})$ is computed using the
eigendecomposition-based method ($K = W\Lambda W^{-1}$, not SVD),
yielding a mean field magnitude $\langle I(\mathbf{x}) \rangle = 8.42$.
The field is elevated during high-volatility regimes and peaks during
crisis transitions, providing a state-space-resolved view of market
non-equilibrium that is inaccessible to global entropy production
estimates alone (Fig.~\ref{fig:irreversibility_field}).

\begin{figure}
\includegraphics[width=\columnwidth]{fig5_irreversibility_field}
\caption{\label{fig:irreversibility_field}Pointwise irreversibility field
$I(\mathbf{x}) = \sum_k \sigma_k |u_k(\mathbf{x}) - v_k(\mathbf{x})|^2$
computed via eigendecomposition of the Koopman matrix.  The field is
elevated during high-volatility regimes and peaks at crisis transitions,
providing state-space-resolved information about non-equilibrium
behavior.  Mean field magnitude: $\langle I \rangle = 8.42$; maximum:
$I_{\max} = 1952$.}
\end{figure}

\subsection{Detailed balance violation}
\label{sec:db_results}

The detailed balance violation metric $\mathcal{D} = 0.73$
(Eq.~\ref{eq:db_metric}), measuring the Frobenius asymmetry of the
Koopman operator, confirms that the dynamics deviate substantially
from reversibility. The ratio
$\|K\|_F / \|K^T\|_F = 1.06$ provides a complementary
measure indicating that the operator is not far from normal
despite the large asymmetric component.

The fluctuation theorem ratio
$\langle e^{-\dot{s}\tau} \rangle = 0.87$ deviates from the
Gallavotti-Cohen prediction of unity (log deviation 0.15),
consistent with non-Gaussian tails in the entropy production
distribution and the approximate nature of the spectral decomposition
at finite mode count~\cite{gaspard2004time}. The complex fraction
(ratio of imaginary to total spectral content) is 0.54,
confirming that oscillatory modes carry more than half of the
total dynamical content.

\subsection{Rolling spectral analysis}
\label{sec:rolling_results}

Rolling-window analysis (window size $W = 500$ trading days,
stride = 5 days) tracks the time evolution of the spectral gap
and entropy production rate over 1563 overlapping windows.
The spectral gap exhibits a concurrent correlation of $r = -0.29$
with the VIX volatility index ($n = 1563$ windows), confirming that
periods of elevated market stress correspond to larger spectral gaps
(faster relaxation dynamics). Cross-correlation
analysis reveals an optimal correlation of $r = -0.30$ at a lag of
13 trading days.
Formal Granger causality testing, with date-aligned spectral gap
and VIX time series, does not reach statistical significance at the
5\% level. The spectral gap therefore captures a \emph{concurrent}
association with realized volatility rather than a predictive lead.
This is consistent with the interpretation that both quantities
respond to the same underlying market state transitions, with the
spectral gap providing a complementary operator-theoretic
characterization of the volatility regime
(Figs.~\ref{fig:spectral_gap_vix}, \ref{fig:rolling_entropy},
\ref{fig:cross_correlation}).

\subsection{Baseline comparison}
\label{sec:baseline_results}

KTND is compared against five baseline methods for regime detection:
a 3-state Gaussian HMM (Baum-Welch EM), truncated DMD ($K=10$ modes),
PCA + $K$-means clustering ($K=3$), GARCH(1,1) conditional volatility
with percentile threshold, and deterministic VIX thresholds
($<20$: low, $20$--$30$: medium, $>30$: crisis). All methods are
evaluated against NBER-dated recession periods using accuracy, precision,
recall, and F1 score (Table~\ref{tab:baselines},
Fig.~\ref{fig:regime_comparison}). The label-to-recession mapping
for all methods is learned on training data only (pre-2018) to prevent
data snooping.

KTND's regime assignments are obtained by fitting a 2-state Gaussian
HMM to the top three Koopman eigenfunctions, leveraging both the
amplitude structure and temporal transition dynamics of the learned
operator. This approach is motivated by the Perron--Frobenius duality:
the dominant eigenfunctions partition state space into metastable sets,
and the HMM captures the Markov transition structure between them.
The VIX threshold baseline achieves the
highest NBER accuracy (0.90) and F1 score (0.37), benefiting from the
strong contemporaneous relationship between volatility levels and
recession dating. The GARCH(1,1) baseline achieves comparable
F1 (0.35) using conditional volatility regime classification.
The HMM baseline achieves high accuracy (0.88) but
low F1 (0.03), while DMD and PCA produce identical regime
assignments with moderate F1 (0.21).

We emphasize that KTND's primary contribution is \emph{not}
regime detection accuracy, where all methods---including the
naive baseline (always predicting expansion)---achieve $\geq 0.82$
accuracy due to the class imbalance (recessions comprise $\sim$8\%
of the sample). Rather, the framework provides unique physical
observables: the entropy production decomposition, irreversibility
field, and fluctuation theorem diagnostics are fundamentally
inaccessible to the baseline methods. The KTND regime assignments
serve as an exploratory application mapping the learned dynamical
structure to economic cycles.

\begin{table}
\caption{\label{tab:baselines}Baseline comparison for NBER recession
detection. Accuracy, precision, recall, and F1 are computed against
NBER-dated recession periods (post-2000). KTND uses a 2-state Gaussian
HMM fitted to the top 3 Koopman eigenfunctions. All label-to-recession
mappings are learned on training data only (pre-2018).}
\begin{ruledtabular}
\begin{tabular}{ldddd}
Method & \multicolumn{1}{c}{Acc.} & \multicolumn{1}{c}{Prec.} &
  \multicolumn{1}{c}{Rec.} & \multicolumn{1}{c}{F1} \\
\hline
KTND (HMM on $\psi_{1\text{--}3}$) & \multicolumn{4}{c}{\textit{(from HMM re-detection run)}} \\
HMM ($K\!=\!3$) & 0.88 & 0.04 & 0.02 & 0.03 \\
DMD ($K\!=\!10$) & 0.83 & 0.16 & 0.28 & 0.21 \\
PCA + $K$-means & 0.83 & 0.16 & 0.28 & 0.21 \\
GARCH(1,1) & 0.82 & 0.26 & 0.50 & 0.35 \\
VIX threshold & 0.90 & 0.35 & 0.38 & 0.37 \\
Naive (majority) & 0.92 & --- & 0.00 & 0.00 \\
\end{tabular}
\end{ruledtabular}
\end{table}

\begin{figure}
\includegraphics[width=\columnwidth]{fig6_regime_comparison}
\caption{\label{fig:regime_comparison}Regime detection comparison.
Accuracy, precision, recall, and F1 score for each method evaluated
against NBER-dated recession periods. All methods achieve $\geq 0.82$
accuracy due to class imbalance ($\sim$8\% recession rate). The VIX
threshold achieves the highest F1 (0.37). KTND's primary contribution
is the non-equilibrium observables (entropy, irreversibility), not
regime detection accuracy.}
\end{figure}

\begin{figure}
\includegraphics[width=\columnwidth]{fig7_training_curves}
\caption{\label{fig:training_curves}Training and validation loss curves
for the univariate SPY model (500 epochs, early stopping patience 50).
The total loss (solid) combines the VAMP-2 score (dominant), orthogonality
regularizer, entropy consistency, and spectral penalty terms. Convergence
is achieved within $\sim$200 epochs; no significant train-validation gap
indicates the absence of overfitting.}
\end{figure}

\begin{figure}
\includegraphics[width=\columnwidth]{fig8_chapman_kolmogorov}
\caption{\label{fig:ck_test}Chapman-Kolmogorov consistency test.
Frobenius norm error $\epsilon_{\mathrm{CK}}(n) = \|[K(\tau)]^n -
K_{\mathrm{direct}}(n\tau)\|_F$ for $n = 2, 3, 4, 5$.  Mean error
$\bar{\epsilon} = 0.129$ across all $n$, indicating approximate Markov
consistency of the learned operator.  See
Sec.~\ref{sec:stat_results} for statistical assessment.}
\end{figure}

\begin{figure}
\includegraphics[width=\columnwidth]{fig9_bootstrap_ci}
\caption{\label{fig:bootstrap_ci}Block-bootstrap 95\% confidence
intervals for eigenvalue magnitudes (200 replicates, block size 20 days).
All 10 modes are well resolved with non-overlapping CIs for the dominant
modes, confirming that the spectral structure is statistically robust.
The leading mode has $|\lambda_1| = 0.39 \pm 0.08$ (bootstrap
mean $\pm$ std).}
\end{figure}

\subsection{Statistical validation results}
\label{sec:stat_results}

We report the outcomes of the five statistical tests described in
Sec.~\ref{sec:stat_validation}, presenting both supportive and
unfavorable results for transparency.

\paragraph{Chapman-Kolmogorov test.}
The mean CK error across $n = 2, 3, 4, 5$ is
$\bar{\epsilon}_{\mathrm{CK}} = 0.129$ (individual errors: 0.134,
0.131, 0.120, 0.133), indicating reasonable but not perfect Markov
consistency. The errors are relatively stable across $n$, suggesting
that deviations are systematic rather than growing. In the Markov
limit one expects $\epsilon_{\mathrm{CK}} \to 0$; the finite residual
likely reflects the approximate nature of the finite-dimensional Koopman
projection and mild non-stationarity over the long training window
(Fig.~\ref{fig:ck_test}).

\paragraph{Bootstrap eigenvalue confidence intervals.}
Block-bootstrap CIs (200 replicates, block size 20 days) confirm that
all 10 eigenvalue magnitudes are well resolved. The leading mode has
$|\lambda_1| = 0.39 \pm 0.08$ with 95\% CI $[0.26, 0.54]$; the
least-resolved mode (mode 9) has $|\lambda_{10}| = 0.08 \pm 0.05$
with 95\% CI $[0.002, 0.18]$.  CIs for the top 5 modes do not overlap,
confirming a robust spectral hierarchy
(Fig.~\ref{fig:bootstrap_ci}).

\paragraph{Permutation test for irreversibility.}
The observed mean irreversibility ($\bar{I} = 4.41$) exceeds the
permutation null distribution (null mean $3.84 \pm 0.41$, 500
permutations) but achieves only marginal significance ($p = 0.056$,
not significant at $\alpha = 0.05$). This borderline result should
be interpreted cautiously: the permutation test destroys temporal
ordering, providing a conservative test of whether the irreversibility
field captures genuine time-asymmetry. The marginal $p$-value may
reflect the test's limited statistical power given the heavy-tailed
distribution of field values.

\paragraph{Ljung-Box residual test.}
All 10 embedding dimensions exhibit strong residual autocorrelation
($p \approx 0$ at 20 lags for all dimensions), indicating that the
Koopman model does not fully capture the temporal dependence in the
data. This is expected: the 10-mode Koopman decomposition provides a
low-rank approximation that captures the dominant spectral structure
but cannot represent all dynamical information in the system. The
residual autocorrelation is consistent with the presence of
higher-order modes beyond the $K = 10$ resolved modes.

\paragraph{KS eigenfunction stability.}
The Kolmogorov-Smirnov test comparing train and test eigenfunction
distributions finds 9 of 10 modes significantly different after
Bonferroni correction ($\alpha_{\mathrm{Bonf}} = 0.005$). Only mode 1
is non-significant ($p = 0.133$). This widespread distributional shift
reflects the non-stationarity of financial markets: the test period
(2020--2025, including COVID crash and rate-hike cycle) exhibits
different dynamics than the training period (1994--2017). The KS result
does not invalidate the Koopman decomposition---which is fit on the
training data---but indicates that the learned eigenfunctions do not
transfer stationarily to the out-of-sample period, consistent with the
rolling-window approach taken in Sec.~\ref{sec:rolling_results}.


% ============================================================================
% VII. ABLATION STUDIES
% ============================================================================
\section{Ablation Studies}\label{sec:ablations}

We conduct 14 systematic ablation studies, each with 10 random seeds,
to isolate the contribution of each model component:

\begin{enumerate}
  \item \textbf{Architecture sweep}: Varying hidden layer widths
    ($[64, 64]$, $[128, 128, 64]$, $[256, 256, 128]$).
  \item \textbf{Mode count sweep}: $K = 3, 5, 10, 20, 50$.
  \item \textbf{Lag sweep}: $\tau = 1, 2, 5, 10, 20$ days.
  \item \textbf{Embedding dimension}: $m = 1, 3, 5, 7, 10$.
  \item \textbf{Dropout}: $p = 0.0, 0.1, 0.2, 0.3, 0.5$.
  \item \textbf{Shared weights}: Reversible vs.\ non-reversible
    architecture.
  \item \textbf{No orthogonality}: $w_2 = 0$.
  \item \textbf{No entropy consistency}: $w_3 = 0$.
  \item \textbf{No spectral penalty}: $w_4 = 0$.
  \item \textbf{No embedding}: Raw returns without delay coordinates.
  \item \textbf{Standardization}: z-score vs.\ robust (median/IQR).
  \item \textbf{Linear features}: No nonlinear embedding.
  \item \textbf{Learning rate}: $\eta = 3\times10^{-4}, 10^{-3}, 3\times10^{-3}$.
\end{enumerate}

Metrics reported include the VAMP-2 score, spectral gap, total
entropy production, and eigenvalue coefficient of variation.
Results for the most informative variants are summarized in
Table~\ref{tab:ablations}.

\begin{table*}
\caption{\label{tab:ablations}Ablation study results (mean $\pm$ std
over 10 seeds).  VAMP-2 is the negative VAMP-2 loss (more negative =
higher score).  Values from 10-seed runs with fixed ablation runner.}
\begin{ruledtabular}
\begin{tabular}{lD{.}{.}{2.2}@{${}\pm{}$}D{.}{.}{1.2}D{.}{.}{1.3}@{${}\pm{}$}D{.}{.}{1.3}D{.}{.}{2.2}@{${}\pm{}$}D{.}{.}{1.2}D{.}{.}{1.2}@{${}\pm{}$}D{.}{.}{1.2}}
Variant & \multicolumn{2}{c}{VAMP-2} & \multicolumn{2}{c}{Spec.\ gap}
  & \multicolumn{2}{c}{Entropy} & \multicolumn{2}{c}{Eig.\ CV} \\
\hline
\textbf{Default} [$128^2\!\times\!64$, $\tau\!=\!5$, $m\!=\!5$]
  & -1.89 & 0.04 & 0.000 & 0.000 & 1.57 & 0.07 & 0.38 & 0.03 \\
\hline
\multicolumn{9}{l}{\textit{Architecture}} \\
\quad $[64, 64]$
  & -1.53 & 0.10 & 0.051 & 0.072 & 1.35 & 0.34 & 0.48 & 0.05 \\
\quad $[256^2\!\times\!128]$
  & -1.94 & 0.15 & 0.116 & 0.089 & 1.71 & 0.08 & 0.34 & 0.06 \\
\hline
\multicolumn{9}{l}{\textit{Mode count $K$}} \\
\quad $K = 3$
  & -0.62 & 0.02 & 0.200 & 0.155 & 0.58 & 0.15 & 0.22 & 0.15 \\
\quad $K = 5$
  & -0.80 & 0.05 & 0.071 & 0.072 & 0.93 & 0.26 & 0.25 & 0.16 \\
\quad $K = 20$
  & -3.65 & 0.12 & 0.169 & 0.044 & 3.20 & 0.25 & 0.64 & 0.03 \\
\quad $K = 50$
  & -12.3 & 0.04 & 0.014 & 0.020 & 6.91 & 0.12 & 0.47 & 0.02 \\
\hline
\multicolumn{9}{l}{\textit{Lag $\tau$ (days)}} \\
\quad $\tau = 1$
  & -9.99 & 0.00 & 0.000 & 0.000 & 34.3 & 6.8 & 0.00 & 0.00 \\
\quad $\tau = 2$
  & -9.99 & 0.00 & 0.000 & 0.000 & 8.97 & 0.85 & 0.00 & 0.00 \\
\quad $\tau = 10$
  & -0.89 & 0.05 & 0.103 & 0.085 & 0.36 & 0.06 & 0.87 & 0.18 \\
\quad $\tau = 20$
  & -0.64 & 0.13 & 0.040 & 0.047 & 0.09 & 0.03 & 0.50 & 0.06 \\
\hline
\multicolumn{9}{l}{\textit{Embedding dimension $m$}} \\
\quad $m = 1$ (no embedding)
  & -0.58 & 0.13 & 0.016 & 0.022 & 1.52 & 0.22 & 0.66 & 0.10 \\
\quad $m = 3$
  & -1.46 & 0.12 & 0.000 & 0.000 & 1.17 & 0.20 & 0.47 & 0.11 \\
\quad $m = 7$
  & -9.99 & 0.00 & 0.000 & 0.000 & 1.27 & 0.14 & 0.00 & 0.00 \\
\hline
\multicolumn{9}{l}{\textit{Dropout $p$}} \\
\quad $p = 0$
  & -1.68 & 0.07 & 0.059 & 0.041 & 1.50 & 0.29 & 0.35 & 0.10 \\
\quad $p = 0.3$
  & -1.32 & 0.02 & 0.068 & 0.058 & 1.73 & 0.56 & 0.81 & 0.13 \\
\hline
\multicolumn{9}{l}{\textit{Other variants}} \\
\quad Linear features
  & -0.26 & 0.00 & 0.000 & 0.000 & 0.96 & 0.08 & 0.54 & 0.09 \\
\quad No orthogonality ($w_2\!=\!0$)
  & -1.95 & 0.03 & 0.050 & 0.039 & 1.50 & 0.07 & 0.39 & 0.03 \\
\quad Shared weights$^\dagger$
  & -1.89 & 0.04 & 0.000 & 0.000 & 1.57 & 0.07 & 0.38 & 0.03 \\
\quad No entropy$^\dagger$
  & -1.89 & 0.04 & 0.000 & 0.000 & 1.57 & 0.07 & 0.38 & 0.03 \\
\end{tabular}
\end{ruledtabular}
\end{table*}

Key findings from the ablation analysis:

\begin{itemize}
  \item \textbf{Lag time is critical}: Short lags ($\tau = 1, 2$) produce
    anomalously high entropy production (34.3 and 9.0, respectively) with
    near-maximal VAMP-2 scores, likely reflecting autocorrelation structure
    rather than genuine Koopman dynamics. At long lags ($\tau = 20, 50$),
    the signal diminishes as modes decay below the noise floor. The default
    $\tau = 5$ achieves the best balance.
  \item \textbf{Embedding is essential}: Without delay embedding ($m = 1$),
    the VAMP-2 score drops from $-1.89$ to $-0.58$, confirming that raw
    returns lack sufficient state information for Koopman analysis. However,
    excessive embedding ($m = 7, 10$) degrades performance, consistent with
    overfitting in high-dimensional delay spaces.
  \item \textbf{Mode count trades resolution for stability}: $K = 3$ is
    too few (VAMP-2 $= -0.62$), while $K = 50$ achieves higher total VAMP-2
    ($-12.3$) but with reduced eigenvalue coefficient of variation (0.47),
    suggesting many nearly degenerate modes. The default $K = 10$ provides
    interpretable spectral structure.
  \item \textbf{Linear features are insufficient}: Replacing the nonlinear
    MLP with a linear projection reduces VAMP-2 to $-0.26$, confirming that
    nonlinear basis functions are essential for capturing the Koopman
    eigenfunction structure.
  \item \textbf{Several ablations are insensitive}: The shared-weight,
    no-entropy-consistency, and no-spectral-penalty variants produce results
    statistically indistinguishable from the default (marked $\dagger$).
    This may reflect either that these components have minimal impact at the
    default configuration or that the ablation mechanism converged to the
    same basin.
\end{itemize}


% ============================================================================
% VIII. DISCUSSION
% ============================================================================
\section{Discussion}\label{sec:discussion}

\subsection{Physical interpretation}

The non-equilibrium Koopman decomposition provides a physically grounded
interpretation of market dynamics that goes beyond traditional statistical
measures. The spectral gap connects to Kramers' barrier-crossing theory,
with market regime transitions analogous to thermally activated escape from
metastable states. The entropy production decomposition reveals which
dynamical timescales drive irreversibility, potentially identifying the
signatures of informed trading, momentum strategies, or regulatory
interventions.

The dominance of complex modes (8 of 10) is noteworthy because it
quantifies a long-suspected but rarely formalized property of financial
markets: the arrow of time is not merely present but is
\emph{multi-scale}, operating through several distinct oscillatory
channels simultaneously. The entropy production decomposition
$\dot{S}_k = \omega_k^2 A_k$ reveals that the fastest oscillatory
modes contribute most to irreversibility (due to the $\omega_k^2$
scaling), suggesting that short-horizon momentum effects and
volatility asymmetry---rather than slow regime
transitions---dominate the market's thermodynamic cost of
non-equilibrium maintenance. This finding is consistent with
microstructure theory, where the continuous flow of information
and the asymmetry between informed and noise
traders~\cite{bouchaud2003theory} sustains probability currents
predominantly at short timescales.

\subsection{Relation to prior work}

Our framework extends the VAMPnet architecture~\cite{mardt2018vampnets} from
reversible molecular dynamics to non-reversible financial dynamics. Previous
applications of Koopman theory to
finance~\cite{mann2016dynamic,kostic2022learning} relied on linear (DMD-based)
methods that cannot capture the nonlinear structure of regime transitions.
Data-driven transfer operator methods~\cite{klus2018data} and sparse
identification approaches~\cite{brunton2016discovering} offer alternatives
but do not provide the thermodynamic observables central to our framework.
The modern Koopman review by Brunton \emph{et al.}~\cite{brunton2022modern}
identifies the extension to non-reversible dynamics as an open direction that
our work directly addresses. KTND learns a nonlinear Koopman embedding that
resolves complex eigenvalues and non-equilibrium mode structure.

The irreversibility field $I(\mathbf{x})$ provides a spatial diagnostic
absent from previous non-equilibrium analyses of financial
data~\cite{jiang2019multifractal,zumbach2009time}, which typically report
only global entropy production estimates or scalar irreversibility indices
without state-space resolution. Zumbach~\cite{zumbach2009time} demonstrated
that financial time series violate time-reversal invariance at multiple
scales using autocorrelation-based measures, and
Lacasa \emph{et al.}~\cite{lacasa2015time} developed visibility-graph
approaches to irreversibility, but neither provides a spectral decomposition
of the violation or pointwise localization---gaps that KTND directly
addresses.

Our entropy production estimation connects to a growing body of work in
Physical Review E on machine-learning-based dissipation
quantification. Otsubo \emph{et al.}~\cite{otsubo2020estimating}
estimated entropy production from short-time fluctuating currents via
neural networks, while Kim \emph{et al.}~\cite{kim2024fdivergence}
improved bounds using $f$-divergence optimization. In biological
systems, Battle \emph{et al.}~\cite{battle2016broken} measured
broken detailed balance at mesoscopic scales, and
Li \emph{et al.}~\cite{li2019quantifying} quantified dissipation
from fluctuating currents. Our spectral decomposition
$\dot{S}_k = \omega_k^2 A_k$ complements these approaches by
resolving entropy production into dynamically interpretable mode
contributions rather than providing a single aggregate estimate.

\subsection{Comparison with econometric approaches}

It is important to articulate precisely what KTND provides beyond
standard econometric tools. Markov regime-switching
models~\cite{hamilton1989new,ang2002regime} identify discrete latent
states but treat transition probabilities as free parameters without
connecting them to a spectral decomposition or irreversibility measure.
GARCH models~\cite{bollerslev1986generalized} capture volatility
clustering through conditional heteroskedasticity but assume a
reversible innovation process (the standardized residuals are i.i.d.\
by construction). Stochastic volatility
models~\cite{heston1993closed} introduce richer dynamics but lack a
framework for decomposing which dynamical timescales contribute to
irreversibility.

KTND provides three capabilities absent from these approaches:
(i) a \emph{spectral decomposition} of irreversibility into
per-mode contributions $\dot{S}_k$, identifying which oscillatory
timescales drive non-equilibrium behavior;
(ii) a \emph{pointwise irreversibility field} $I(\mathbf{x})$ that
localizes broken detailed balance in state space, enabling
identification of specific market conditions where non-equilibrium
effects concentrate; and
(iii) a \emph{regime persistence bound} $T_{\mathrm{persist}} \geq
1/\Delta$ derived from the Koopman spectral gap, providing a
theoretically grounded (rather than empirically fitted) timescale for
regime duration.

These are not improvements in prediction accuracy---which we do
not claim---but rather new \emph{observables} that characterize
the dynamical structure of the market in a physically interpretable
language. The regime detection accuracy comparison
(Sec.~\ref{sec:baseline_results}) demonstrates that KTND is
competitive with dedicated regime-switching models, while
providing the additional non-equilibrium diagnostics as a bonus.

\subsection{Limitations}

Several limitations warrant discussion:
\begin{enumerate}
  \item \textbf{Stationarity assumption}: The Koopman framework assumes a
    stationary process. Financial markets are at best locally stationary;
    the rolling-window analysis partially addresses this but does not
    provide a non-stationary Koopman theory.
  \item \textbf{Lag time selection}: The choice of $\tau$ affects the
    timescales captured. Our ablation study explores sensitivity, but an
    adaptive $\tau$ remains an open problem.
  \item \textbf{Entropy decomposition approximation}: The per-mode
    entropy production $\dot{S}_k = \omega_k^2 A_k$ is derived under
    a weakly dissipative assumption~\cite{gaspard2004time}. For strongly
    non-equilibrium regimes (e.g., crisis periods where $\omega_k \sim
    \gamma_k$), the decomposition may underestimate the true per-mode
    contributions. The entropy consistency loss provides an empirical
    check, but a fully non-perturbative spectral decomposition remains
    an open theoretical problem.
  \item \textbf{KDE entropy estimation}: Gaussian KDE-based entropy
    production estimates suffer from the curse of dimensionality in
    high-dimensional joint spaces~\cite{kraskov2004estimating}. While our
    univariate analysis ($d=1$, 10D joint space) is tractable, the
    multivariate case ($d=11$, 110D joint space) would require
    $k$-nearest-neighbor estimators for reliable density
    estimation~\cite{kozachenko1987sample}. The block-bootstrap confidence
    intervals quantify finite-sample uncertainty but not systematic
    dimensionality bias.
  \item \textbf{Spectral gap bound}: The regime persistence bound
    $T_{\mathrm{persist}} \geq 1/\Delta$ is conservative for
    non-reversible dynamics, where non-conservative probability currents
    can accelerate mixing beyond what the spectral gap
    predicts~\cite{hwang2005accelerating}.
  \item \textbf{Finite-sample effects}: Covariance estimation from finite
    data introduces bias, particularly for high-dimensional multivariate
    analyses.
  \item \textbf{Causal interpretation}: While the spectral decomposition
    is informative, it does not establish causal mechanisms for
    irreversibility.
  \item \textbf{VIX lead-lag relationship}: The spectral gap exhibits
    a concurrent anti-correlation with VIX ($r = -0.29$) but formal
    Granger causality testing does not reach statistical significance.
    The spectral gap captures a concurrent association with realized
    volatility rather than a predictive lead, consistent with both
    quantities responding to the same underlying market state.
  \item \textbf{Regime detection accuracy}: The KTND eigenfunction-based
    regime detection is presented as an exploratory application. All
    methods achieve $\geq 0.82$ NBER accuracy due to the severe class
    imbalance ($\sim$8\% recession rate), and the naive baseline
    (always predicting expansion) achieves 0.92 accuracy.
    KTND's primary contribution is the non-equilibrium observables,
    not regime classification performance.
  \item \textbf{Single-market scope}: All empirical results are drawn
    from U.S.\ equity markets. We do not claim universality of the
    specific spectral structure or entropy production magnitudes across
    different asset classes, geographic markets, or time periods.
    Whether the 8-of-10 complex mode ratio or the specific entropy
    production rate are universal features of developed equity markets
    or artifacts of U.S.\ market microstructure remains an open
    empirical question requiring multi-market validation.
    Following the critique of
    Gallegati \emph{et al.}~\cite{gallegati2006worrying}, we explicitly
    refrain from claims of universality and present our results as
    characterizing a \emph{specific} complex system rather than
    establishing general laws.
\end{enumerate}

\subsection{Future directions}

Extensions include (i) time-dependent Koopman operators for non-stationary
dynamics, (ii) incorporation of exogenous variables (monetary policy,
sentiment) as control inputs, (iii) application to high-frequency data
where non-equilibrium effects should be more pronounced, (iv) extension
to option-implied state spaces, (v) replacing the KDE-based entropy
production estimator with $k$-nearest-neighbor methods based on the
Kozachenko-Leonenko framework~\cite{kozachenko1987sample,kraskov2004estimating},
which provide dimension-robust density ratio estimation suitable for
high-dimensional multivariate analyses, and (vi) multi-market validation
across international equity indices, foreign exchange, and fixed income
to assess which features of the non-equilibrium spectral structure
are asset-class-specific and which may reflect more general properties
of driven complex systems with heterogeneous interacting agents.


% ============================================================================
% IX. CONCLUSION
% ============================================================================
\section{Conclusion}\label{sec:conclusion}

We have introduced KTND, a spectral framework for quantifying
irreversibility in driven complex systems that combines Koopman operator
theory with non-equilibrium statistical mechanics. The dual-lobe neural
architecture detects broken detailed balance through independent
parameterization of left and right eigenfunction approximations, yielding
complex Koopman eigenvalues that encode oscillatory probability currents.
The resulting spectral entropy production decomposition
$\dot{S}_k = \omega_k^2 A_k$ and pointwise irreversibility field
$I(\mathbf{x})$ provide physically interpretable diagnostics of
non-equilibrium behavior that are inaccessible to reversible methods.

Validation on non-reversible Langevin dynamics confirms recovery of
Kramers' escape rates from the learned spectrum. Application to U.S.\
equity markets---treated as a prototypical driven complex system with
heterogeneous interacting agents---reveals that 8 of 10 Koopman modes
are complex, the entropy production rate is significantly positive
($\dot{S} = 51.2$, 95\% CI $[48.9, 52.3]$), and the spectral gap
yields a regime persistence bound consistent with NBER-dated recession
timing. These results quantify the multi-scale irreversibility of
market dynamics and demonstrate that the framework provides new
observables---spectral entropy decomposition, state-space-resolved
irreversibility, and operator-theoretic regime bounds---that complement
existing econometric and machine learning approaches.


% ============================================================================
% ACKNOWLEDGMENTS
% ============================================================================
\begin{acknowledgments}
  The author thanks the open-source communities behind PyTorch, NumPy,
  SciPy, and scikit-learn for making this work possible.
\end{acknowledgments}

\section*{Data availability}
All market data used in this study are publicly available from Yahoo
Finance (\texttt{yfinance} Python package). NBER recession dates are
obtained from the National Bureau of Economic Research public database.

\section*{Code availability}
The complete implementation, including training code, analysis scripts,
configuration files, and the full test suite (132 unit tests), is
available at \url{https://github.com/keshavkrishnan08/kind_finance}.


% ============================================================================
% SUPPLEMENTAL FIGURES
% ============================================================================
\appendix
\section{Supplemental Figures}\label{app:supplemental}

\begin{figure}
\includegraphics[width=\columnwidth]{supplemental/figS1_eigenvalue_bar}
\caption{\label{fig:eigenvalue_bar}Eigenvalue magnitude bar chart for all
10 Koopman modes, sorted by magnitude.  The clear separation between the
leading modes ($|\lambda| > 0.6$) and the trailing modes
($|\lambda| < 0.5$) supports the low-rank spectral decomposition.}
\end{figure}

\begin{figure}
\includegraphics[width=\columnwidth]{supplemental/figS3_rolling_entropy}
\caption{\label{fig:rolling_entropy}Rolling-window entropy production
time series (window size $W = 500$ days, stride 5 days).  Entropy
production spikes during crisis periods, reflecting increased dynamical
irreversibility.  The rolling analysis covers 1563 overlapping windows
from 1994 to 2025.}
\end{figure}

\begin{figure}
\includegraphics[width=\columnwidth]{supplemental/figS4_cross_correlation}
\caption{\label{fig:cross_correlation}Cross-correlation function between
the spectral gap and VIX index as a function of lag (days).  The minimum
correlation ($r = -0.29$) occurs at lag $= -8$ days, indicating that
spectral gap changes lead VIX movements.  Negative lags correspond to
the spectral gap leading.}
\end{figure}

\begin{figure}
\includegraphics[width=\columnwidth]{supplemental/figS5_eigenfunction_distributions}
\caption{\label{fig:eigfunc_distributions}Train (blue) and test (orange)
eigenfunction distributions for all 10 modes.  KS tests reveal 9 of 10
modes are significantly different ($p < 0.005$ after Bonferroni
correction), reflecting non-stationarity between the training period
(1994--2017) and the test period (2020--2025, including COVID and
rate-hike regimes).}
\end{figure}

\begin{figure}
\includegraphics[width=\columnwidth]{supplemental/figS6_permutation_null}
\caption{\label{fig:permutation_null}Permutation test for irreversibility.
The observed mean irreversibility ($\bar{I} = 4.41$, red dashed line)
is compared against the null distribution obtained by permuting temporal
indices (500 permutations, null mean $3.84 \pm 0.41$).  The one-sided
$p$-value is 0.056, marginally significant.}
\end{figure}

\begin{figure*}
\includegraphics[width=\textwidth]{supplemental/figS7_baseline_regimes}
\caption{\label{fig:baseline_regimes}Regime time series for all baseline
methods (HMM, DMD, PCA + $K$-means, VIX threshold) compared against
NBER recession shading (gray).  Each method assigns regime labels to
every trading day; the time series visualize regime persistence and
transition timing across the full sample period.}
\end{figure*}

\begin{figure}
\includegraphics[width=\columnwidth]{supplemental/figS8_singular_values}
\caption{\label{fig:singular_values}Singular value spectrum of the
whitened Koopman matrix $K$.  The 10 singular values range from
$\sigma_1 = 0.90$ to $\sigma_{10} = 0.14$, all satisfying the
contraction bound $\sigma_k \leq 1$.  The gradual decay suggests a
well-resolved spectral hierarchy without a sharp cutoff.}
\end{figure}


% ============================================================================
% BIBLIOGRAPHY
% ============================================================================
\bibliography{references}

\end{document}
